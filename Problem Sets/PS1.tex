% Options for packages loaded elsewhere
\PassOptionsToPackage{unicode}{hyperref}
\PassOptionsToPackage{hyphens}{url}
%
\documentclass[
]{article}
\usepackage{lmodern}
\usepackage{amssymb,amsmath}
\usepackage{ifxetex,ifluatex}
\ifnum 0\ifxetex 1\fi\ifluatex 1\fi=0 % if pdftex
  \usepackage[T1]{fontenc}
  \usepackage[utf8]{inputenc}
  \usepackage{textcomp} % provide euro and other symbols
\else % if luatex or xetex
  \usepackage{unicode-math}
  \defaultfontfeatures{Scale=MatchLowercase}
  \defaultfontfeatures[\rmfamily]{Ligatures=TeX,Scale=1}
\fi
% Use upquote if available, for straight quotes in verbatim environments
\IfFileExists{upquote.sty}{\usepackage{upquote}}{}
\IfFileExists{microtype.sty}{% use microtype if available
  \usepackage[]{microtype}
  \UseMicrotypeSet[protrusion]{basicmath} % disable protrusion for tt fonts
}{}
\makeatletter
\@ifundefined{KOMAClassName}{% if non-KOMA class
  \IfFileExists{parskip.sty}{%
    \usepackage{parskip}
  }{% else
    \setlength{\parindent}{0pt}
    \setlength{\parskip}{6pt plus 2pt minus 1pt}}
}{% if KOMA class
  \KOMAoptions{parskip=half}}
\makeatother
\usepackage{xcolor}
\IfFileExists{xurl.sty}{\usepackage{xurl}}{} % add URL line breaks if available
\IfFileExists{bookmark.sty}{\usepackage{bookmark}}{\usepackage{hyperref}}
\hypersetup{
  pdftitle={Math 15910: Problem Set 1},
  pdfauthor={Underland, Jake},
  hidelinks,
  pdfcreator={LaTeX via pandoc}}
\urlstyle{same} % disable monospaced font for URLs
\usepackage[margin=1in]{geometry}
\usepackage{graphicx,grffile}
\makeatletter
\def\maxwidth{\ifdim\Gin@nat@width>\linewidth\linewidth\else\Gin@nat@width\fi}
\def\maxheight{\ifdim\Gin@nat@height>\textheight\textheight\else\Gin@nat@height\fi}
\makeatother
% Scale images if necessary, so that they will not overflow the page
% margins by default, and it is still possible to overwrite the defaults
% using explicit options in \includegraphics[width, height, ...]{}
\setkeys{Gin}{width=\maxwidth,height=\maxheight,keepaspectratio}
% Set default figure placement to htbp
\makeatletter
\def\fps@figure{htbp}
\makeatother
\setlength{\emergencystretch}{3em} % prevent overfull lines
\providecommand{\tightlist}{%
  \setlength{\itemsep}{0pt}\setlength{\parskip}{0pt}}
\setcounter{secnumdepth}{-\maxdimen} % remove section numbering

\title{Math 15910: Problem Set 1}
\author{Underland, Jake}
\date{2021-08-09}

\begin{document}
\maketitle

{
\setcounter{tocdepth}{2}
\tableofcontents
}
\hypertarget{exercise-1}{%
\section{Exercise 1}\label{exercise-1}}

\hypertarget{problem-1.2.2}{%
\subsection{Problem 1.2.2}\label{problem-1.2.2}}

\textbf{\emph{If \(A\) and \(B\) are sets, show that \(A = B\) if and
only if \(A \subseteq B\) and \(B \subseteq A\).}}

From \(A=B\) we know that for any \(x\),
\[\text{if } x \in A, \text{ then } x \in B \text{ ... (1)}\] and,
conversely,
\[\text{if } x \in B, \text{ then } x \in A \text{ ... (2)}\] From (1),
by definition we obtain \(A \subseteq B\) and from (2), we obtain
\(B \subseteq A\). Thus, we have proven
\[ A = B \implies A \subseteq B \text{ and }B \subseteq A \text{ ... (3)}\]
Now, suppose \(A \subseteq B\) and \(B \subseteq A\). Then, by
definition, \(\text{if } x \in A, \text{ then } x \in B\), and
\(\text{if } x \in B, \text{ then } x \in A\). This is equivalent to
saying \(A = B\). Thus, we have proven
\[A \subseteq B \text{ and }B \subseteq A \implies A = B  \text{ ... (4)}\]
From (3) and (4), we obtain the following:
\[ A = B \iff A \subseteq B \text{ and }B \subseteq A \] Therefore,
\(A = B\) if and only if \(A \subseteq B\) and \(B \subseteq A\).
\hfill \(\Box\)

\hypertarget{problem-1.2.3}{%
\subsection{Problem 1.2.3}\label{problem-1.2.3}}

\textbf{\emph{Suppose that \(A, B, \text{and } C\) are sets. If
\(A \subseteq B\) and \(B\subseteq C\), show that \(A \subseteq C\).}}

If \(A \subseteq B\), it follows that for any \(x\),
\(\text{if } x \in A, \text{ then } x \in B\). If \(B \subseteq C\), it
follows that \(\text{if } x \in B, \text{ then } x \in C\). Suppose that
\(x \in A\). Then, since \(A \subseteq B\), \(x \in B\). Furthermore, if
\(x \in B\), since \(B \subseteq C\), \(x \in C\). Thus, for any
\(x \in A\) we have \(x \in C\), and we have proven that if
\(A \subseteq B\) and \(B\subseteq C\), then \(A \subseteq C\).
\hfill \(\Box\)

\hypertarget{exercise-2}{%
\section{Exercise 2}\label{exercise-2}}

\hypertarget{problem-1.3.9}{%
\subsection{Problem 1.3.9}\label{problem-1.3.9}}

\hypertarget{ii.-prove-a-cap-b-cap-c-a-cap-b-cap-c}{%
\subsubsection{\texorpdfstring{ii. \emph{Prove
\(A \cap (B \cap C) = (A \cap B) \cap C\)}}{ii. Prove A \textbackslash cap (B \textbackslash cap C) = (A \textbackslash cap B) \textbackslash cap C}}\label{ii.-prove-a-cap-b-cap-c-a-cap-b-cap-c}}

Take \(x \in X\). Then, by definition of intersection,
\[x \in A \cap (B \cap C) \iff x \in A \text{ and }x \in B \cap C\]
Furthermore, \[x \in (B \cap C) \iff x \in B \text{ and }x \in C\] Thus,
we have\\
\[ x \in A \cap (B \cap C) \iff x \in A \text{ and }x \in B \text{ and }x \in C \text{ ... (1)}\]
By the definition of intersection,
\[ x \in A \text{ and }x \in B \iff x \in (A \cap B) \text{ ... (2)}\]
From (1) and (2), we have
\[x \in A \cap (B \cap C) \iff x \in (A \cap B) \text{ and }x \in C\] By
the definition of intersection,
\[x \in (A \cap B) \text{ and }x \in C \iff x \in (A \cap B) \cap C\]
Therefore, \(x \in A \cap (B \cap C) \iff x \in (A \cap B) \cap C\) and
we have proven that \(A \cap (B \cap C) = (A \cap B) \cap C\), or in
other words, associativity exists for intersection. \hfill \(\Box\)

\hypertarget{ix.-prove-a-triangle-b-varnothing-iff-a-b}{%
\subsubsection{\texorpdfstring{ix. \emph{Prove
\(A \triangle B = \varnothing \iff A = B\)}}{ix. Prove A \textbackslash triangle B = \textbackslash varnothing \textbackslash iff A = B}}\label{ix.-prove-a-triangle-b-varnothing-iff-a-b}}

Suppose \(A \triangle B = \varnothing\). Then by definition of symmetric
difference,\\
\[A \triangle B = (A \backslash B) \cup (B \backslash A) = \varnothing\]
Note that if the union of two sets is an empty set, then the two sets
must also be empty. This is clear from the definition of union
\(A \cup B = \{x \in X | x \in A \text{ or } x \in B \}\). Suppose that
\(A\) or \(B\), or both, is a nonempty set. Then, there exists at least
one \(\{x \in X | x \in A \text{ or } x \in B \}\). Thus, \(A \cup B\)
would be nonempty also. Since we have proven the contrapositive, it
holds that if the union of two sets is an empty set, then the two sets
must also be empty.\\
From here it follows that
\[(A \backslash B) \cup (B \backslash A) = \varnothing \\ \implies (A \backslash B) = \varnothing \text{ and }(B \backslash A ) = \varnothing\]
Take \((A \backslash B) = \varnothing\). Suppose there is an element
\(x\) such that \(x \in A\) and \(x \notin B\). By definition,
\(x \in A \text{ and }x \notin B \implies x \in A \backslash B\), but
this contradicts with \((A \backslash B) = \varnothing\). Therefore,
such an \(x\) does not exist, meaning for all \(x \in A\), \(x \in B\).
Thus, we obtain \(A \subseteq B\).\\
Now, take \((B \backslash A) = \varnothing\). Through similar reasoning,
we obtain \(B \subseteq A\).\\
Since \(A \subseteq B\) and \(B \subseteq A\), \(A = B\). We have thus
proven that \(A \triangle B = \varnothing \implies A = B\).\\
Now suppose \(A = B\). From earlier proofs, we know that this implies
\(A \subseteq B\) and \(B \subseteq A\). From the definition of
\(A \subseteq B\), we know that for all \(x \in A\), \(x \in B\) and can
thus infer that \((A \backslash B) = \varnothing\). Similarly, we can
infer that \((B \backslash A) = \varnothing\). Since
\[(A \backslash B) = (B \backslash A) = \varnothing \\ \implies(A \backslash B) \cup (B \backslash A) = \varnothing \\ \implies A \triangle B = \varnothing\]
Thus, we have proven that
\(A = B\implies A \triangle B = \varnothing\).\\
From the above, we have shown
\(A \triangle B = \varnothing \iff A = B\). \hfill \(\Box\)

\hypertarget{xi.-prove-a-cup-b-cap-c-a-cup-b-cap-a-cup-c}{%
\subsubsection{\texorpdfstring{xi. \emph{Prove
\(A \cup (B \cap C) = (A \cup B) \cap (A \cup C)\)}}{xi. Prove A \textbackslash cup (B \textbackslash cap C) = (A \textbackslash cup B) \textbackslash cap (A \textbackslash cup C)}}\label{xi.-prove-a-cup-b-cap-c-a-cup-b-cap-a-cup-c}}

Take \(x \in X\). Then \[x \in A \cup (B \cap C)\] By definition of
union, \[\iff x \in A \text{ or } x \in (B \cap C)\] By definition of
intersection,
\[ \iff x \in A \text{ or } (x \in B \text{ and }x \in C)\]
\[ \iff (x \in A \text{ or } x \in B) \text{ and } (x \in A \text{ or } x \in C)\]
By definition of union,
\[\iff (x \in A \cup B) \text{ and } (x \in A \cup C)\] By definition of
intersection, \[\iff x \in (A \cup B) \cap (A \cup C)\] Thus, we have
proven \(A \cup (B \cap C) = (A \cup B) \cap (A \cup C)\).

\hfill \(\Box\)

\hypertarget{xii.-prove-a-cap-bmathsfc-amathsfc-cup-bmathsfc}{%
\subsubsection{\texorpdfstring{xii. \emph{Prove
\((A \cap B)^\mathsf{c} = A^\mathsf{c} \cup B^\mathsf{c}\)}}{xii. Prove (A \textbackslash cap B)\^{}\textbackslash mathsf\{c\} = A\^{}\textbackslash mathsf\{c\} \textbackslash cup B\^{}\textbackslash mathsf\{c\}}}\label{xii.-prove-a-cap-bmathsfc-amathsfc-cup-bmathsfc}}

Let \(x \in X\). Then \[x \in (A \cap B)^\mathsf{c}\] By definition of
complements, \[\iff x \in X \backslash (A \cap B)\] By definition of
difference,
\[\iff x \notin (A \cap B) \text{ ... (since } x \in X \text{ is already given)}\]
\(x \notin (A \cap B)\) is the negation of \(x \in (A \cap B)\), which
is by definition tantamount to \(x \in A \text{ and } x \in B\). It
follows that the negation would be
\(x \notin A \text{ or } x \notin B\). Thus,
\[x \notin (A \cap B) \iff x \notin A \text{ or } x \notin B\] By
definition of difference,
\[\iff x \in X \backslash A \text{ or } x \in X \backslash B\] By
definition of complements,
\[\iff x \in A^\mathsf{c} \text{ or } x \in B^\mathsf{c}\] By definition
of union, \[\iff x \in A^\mathsf{c} \cup B^\mathsf{c}\] Therefore, we
have proven that
\((A \cap B)^\mathsf{c} = A^\mathsf{c} \cup B^\mathsf{c}\).

\hfill \(\Box\)

\hypertarget{exercise-3}{%
\section{Exercise 3}\label{exercise-3}}

\textbf{\emph{Express each of the following statement as a conditional
state- ment in ''if-then'' form. For each statement, write the negation,
the contrapositive and the converse. Your answers should use clear
English, not logical symbols.}}

\hypertarget{a-every-odd-number-is-prime.}{%
\subsection{(a) Every odd number is
prime.}\label{a-every-odd-number-is-prime.}}

\begin{itemize}
\tightlist
\item
  if - then: If a number is odd, then it is prime.\\
\item
  negation: A number is odd and it is not prime.\\
\item
  contrapositive: If a number is not prime, then it is not odd.\\
\item
  converse: If a number is prime, then it is odd.
\end{itemize}

\hypertarget{b-passing-the-test-requires-solving-all-the-problems.}{%
\subsection{(b) Passing the test requires solving all the
problems.}\label{b-passing-the-test-requires-solving-all-the-problems.}}

\begin{itemize}
\tightlist
\item
  if - then: If one is to pass the test, then one must solve all the
  problems.\\
\item
  negation: Someone passed the test but did not solve all the
  problems.\\
\item
  contrapositive: If one does not solve all the problems, then they will
  not pass the test.\\
\item
  converse: If one solves all the problems, then they will pass the
  test.
\end{itemize}

\hypertarget{c-being-first-in-line-guarantees-getting-a-good-seat.}{%
\subsection{(c) Being first in line guarantees getting a good
seat.}\label{c-being-first-in-line-guarantees-getting-a-good-seat.}}

\begin{itemize}
\tightlist
\item
  if - then: If one is first in line, then they are guaranteed a good
  seat.\\
\item
  negation: Someone was first in line but was not guaranteed a good
  seat.\\
\item
  contrapositive: If one is not guaranteed a good seat, then they are
  not first in line.\\
\item
  converse: If one is guaranteed a good seat, then they are first in
  line.
\end{itemize}

\end{document}
