% Options for packages loaded elsewhere
\PassOptionsToPackage{unicode}{hyperref}
\PassOptionsToPackage{hyphens}{url}
%
\documentclass[
]{article}
\usepackage{lmodern}
\usepackage{amssymb,amsmath}
\usepackage{ifxetex,ifluatex}
\ifnum 0\ifxetex 1\fi\ifluatex 1\fi=0 % if pdftex
  \usepackage[T1]{fontenc}
  \usepackage[utf8]{inputenc}
  \usepackage{textcomp} % provide euro and other symbols
\else % if luatex or xetex
  \usepackage{unicode-math}
  \defaultfontfeatures{Scale=MatchLowercase}
  \defaultfontfeatures[\rmfamily]{Ligatures=TeX,Scale=1}
\fi
% Use upquote if available, for straight quotes in verbatim environments
\IfFileExists{upquote.sty}{\usepackage{upquote}}{}
\IfFileExists{microtype.sty}{% use microtype if available
  \usepackage[]{microtype}
  \UseMicrotypeSet[protrusion]{basicmath} % disable protrusion for tt fonts
}{}
\makeatletter
\@ifundefined{KOMAClassName}{% if non-KOMA class
  \IfFileExists{parskip.sty}{%
    \usepackage{parskip}
  }{% else
    \setlength{\parindent}{0pt}
    \setlength{\parskip}{6pt plus 2pt minus 1pt}}
}{% if KOMA class
  \KOMAoptions{parskip=half}}
\makeatother
\usepackage{xcolor}
\IfFileExists{xurl.sty}{\usepackage{xurl}}{} % add URL line breaks if available
\IfFileExists{bookmark.sty}{\usepackage{bookmark}}{\usepackage{hyperref}}
\hypersetup{
  pdftitle={Math 15910: Problem Set 8},
  pdfauthor={Underland, Jake},
  hidelinks,
  pdfcreator={LaTeX via pandoc}}
\urlstyle{same} % disable monospaced font for URLs
\usepackage[margin=1in]{geometry}
\usepackage{graphicx,grffile}
\makeatletter
\def\maxwidth{\ifdim\Gin@nat@width>\linewidth\linewidth\else\Gin@nat@width\fi}
\def\maxheight{\ifdim\Gin@nat@height>\textheight\textheight\else\Gin@nat@height\fi}
\makeatother
% Scale images if necessary, so that they will not overflow the page
% margins by default, and it is still possible to overwrite the defaults
% using explicit options in \includegraphics[width, height, ...]{}
\setkeys{Gin}{width=\maxwidth,height=\maxheight,keepaspectratio}
% Set default figure placement to htbp
\makeatletter
\def\fps@figure{htbp}
\makeatother
\setlength{\emergencystretch}{3em} % prevent overfull lines
\providecommand{\tightlist}{%
  \setlength{\itemsep}{0pt}\setlength{\parskip}{0pt}}
\setcounter{secnumdepth}{-\maxdimen} % remove section numbering

\title{Math 15910: Problem Set 8}
\author{Underland, Jake}
\date{2021-08-09}

\begin{document}
\maketitle

{
\setcounter{tocdepth}{2}
\tableofcontents
}
\hypertarget{exercise-1}{%
\section{Exercise 1}\label{exercise-1}}

\hypertarget{problem-3.2.9}{%
\subsection{Problem 3.2.9}\label{problem-3.2.9}}

\textit{Suppose $\lim _{n \to \infty} x_n = x $ and $\lim _{n \to \infty} y_n = y $}

\begin{enumerate}
\item \textit{Show that $\lim _{n \to \infty} (x_n + y_n) = x + y$}  
  
Proof. From the supposition, we know that for all $\epsilon >0$ there exists an $N \in \mathbb{N}$ such that for all $n>N$,
\[\begin{aligned}|x_n - x| &< \epsilon  \\ |y_n - y| &< \epsilon\end{aligned}\]
Let $\epsilon = \frac{\epsilon'}{2}$. Then, 
\[\begin{aligned}|x_n - x| &< \frac{\epsilon'}{2} \\ |y_n - y| &< \frac{\epsilon'}{2}\end{aligned}\] 
From triangle inequality, 
\[\begin{aligned}|x_n + y_n - x - y| \leq |x_n - x| + |y_n - y| &< \frac{\epsilon'}{2}+ \frac{\epsilon'}{2} \\ |x_n + y_n - (x + y)|&< \epsilon' \end{aligned}\]
This holds for all $\epsilon' > 0$. Thus, $x + y$ is the limit. 

\hfill $\Box$  
 
\item \textit{Show that if for all $n \in \mathbb{N} y_n ̸\neq 0$ and $y \neq 0$, then $\lim_{n\to \infty}(\frac{1}{y_n}) = \frac{1}{y}$.}  
  
Proof. Since $\lim _{n \to \infty} y_n = y$, it is obvious that $\lim _{n \to \infty} |y_n| = |y|$. Then, for all $\epsilon > 0$, there exists an $N$ such that for all $n > N$, $|y_n|$ becomes arbitrarily close to $|y|$. Thus, we can say with certainty that there exists $N_1$ such that for all $n>N_1$, $|y_n| > \frac{|y|}{2}$. This implies $\frac{1}{|y_n|} < \frac{2}{|y|}$.  
Further, let $\epsilon = \epsilon' \frac{|y|^2}{2}$. Then, there also exists $N_2$ such that for all $n > N_2$, $|y_n - y| < \epsilon$.  
Now, let us set $N=\max \{ N_1, N_2\}$. Then, 
\[\begin{aligned}&|\frac{1}{y_n} - \frac{1}{y}| 
\\ = &|\frac{y - y_n}{y_n y}| 
\\ = &\frac{|y_n - y|}{|y_n||y|} 
\\ < &\epsilon' \frac{|y|^2}{2} \cdot \frac{2}{|y|} \cdot \frac{1}{|y|}
\\ = &\epsilon'
\end{aligned}\]
Therefore, for all $\epsilon'>0$, there exists $N \in \mathbb{N}$ such that for all $n > N$, $|\frac{1}{y_n} - \frac{1}{y}| < \epsilon'$. Thus,  $\lim_{n\to \infty}(\frac{1}{y_n}) = \frac{1}{y}$.

\hfill $\Box$ 
\end{enumerate}

\hypertarget{exercise-2}{%
\section{Exercise 2}\label{exercise-2}}

\textit{Let $m \in \mathbb{N}$. Prove that $(x_n)_{n \in \mathbb{N}}$ converges iff $(x_{m+n})_{n \in \mathbb{N}}$ converges. Moreover, show that $\lim_{n\to\infty} x_n = \lim_{n\to\infty} x_{m+n}$}

~~~~Proof. If \((x_n)_{n \in \mathbb{N}}\) converges to \(x\), then we
know that for all \(\epsilon >0\) there exists an \(N \in \mathbb{N}\)
such that for all \(n>N\), \(|x_n - x|<\epsilon\). Since \(m + n > n\),
\((x_{m+n})\) also converges to the same limit \(x\).\\
\hspace*{0.333em}\hspace*{0.333em}\hspace*{0.333em}\hspace*{0.333em}If
\((x_{m+n})_{n \in \mathbb{N}}\) converges to \(x\), then for all
\(\epsilon >0\) there exists an \(N_1 \in \mathbb{N}\) such that for all
\(m + n>N_1\), \(|x_{m+n} - x|<\epsilon\). Then, take \(N_2 = N_1 + m\).
For all \(n + m> N_2, \space n+m > N_1 + m > N_1\). Furthermore,
\(n+m > N_1 + m \implies n > N_1\). Thus, \(|x_{n} - x|<\epsilon\). Such
an \(N_2\) exists for all \(\epsilon\), given that
\((x_{m+n})_{n \in \mathbb{N}}\) converges. Thus,
\((x_n)_{n \in \mathbb{N}}\) converges iff
\((x_{m+n})_{n \in \mathbb{N}}\), and their limits are equivalent.

\hfill \(\Box\)

\hypertarget{exercise-3}{%
\section{Exercise 3}\label{exercise-3}}

\textit{Show that $(x_n)_{n\in \mathbb{N}}$ converges to $L$ if and only if every subsequence of $(x_n)_{n\in \mathbb{N}}$ converges to $L$}

~~~~Proof. Let \(b_n\) be a subsequence of \(x_n\). Note that for all
\(x_n\), \(b_n = x_m \text{ for } m > n\). By definition of convergence,
for all \(\epsilon >0\) there exists an \(N \in \mathbb{N}\) such that
for all \(n>N\), \(|x_n - L|<\epsilon\). Then, since
\(b_n = x_m \text{ where } m > n\), \(|b_n - L|<\epsilon\), and thus
\(b_n\) converges to \(L\). ~~~~If every subsequence of
\((x_n)_{n\in \mathbb{N}}\) converges to \(L\), since a sequence is a
subsequence of itself, it follows that \((x_n)_{n\in \mathbb{N}}\)
converges to \(L\).

\hfill \(\Box\)

\hypertarget{exercise-4}{%
\section{Exercise 4}\label{exercise-4}}

\textit{Prove directly from the definition that $a_n=\frac{n + 2}{2n + 1}$ is Cauchy.}

Proof. Let \(N = \frac{3}{2\epsilon}\). Then, for \(m, n > N\),
\[\begin{aligned}|\frac{n+2}{2n+1} - \frac{m+2}{2m+1}| &= |\frac{3m - 3n}{(2n+1)(2m+1)}| 
\\ &=|\frac{3n - 3m}{(2n+1)(2m+1)}| \\
&\leq |\frac{3n}{(2n+1)(2m+1)}| + |\frac{3m}{(2n+1)(2m+1)}| \\
&\leq |\frac{3n}{(2n2m)}| + |\frac{3m}{(2n2m)}| \\
&= |\frac{3}{4m}| + |\frac{3}{4n}| \\
&< |\frac{\epsilon}{2}| + |\frac{\epsilon}{2}| \\
&= \epsilon
\end{aligned}\] Thus, \(a_n\) is Cauchy.

\hfill \(\Box\)

\hypertarget{exercise-5}{%
\section{Exercise 5}\label{exercise-5}}

\hypertarget{problem-3.6.13}{%
\subsection{Problem 3.6.13}\label{problem-3.6.13}}

\textit{Prove that every Cauchy sequence in $\mathbb{R}$ is bounded.}

~~~~Proof. Since the sequence is Cauchy, for all \(\epsilon > 0\), there
exists \(N\) such that for all \(m, n > N\), \(|a_n - a_m| < \epsilon\).
Let \(\epsilon = \epsilon_0\) and \(N = N_0\) be the \(N\) such that for
all \(m, n > N_0\), \(|a_n - a_m| < \epsilon_0\). Furthermore, let
\(m=m_0>N_0\). Then, from triangle inequality,
\[\begin{aligned}|a_n -a_{m_0}| &<\epsilon_0 \\
\implies |a_n -a_{m_0}| + |a_{m_0}| &<\epsilon_0 + |a_{m_0}| \\
\text{From triangle inequality,} \\
|a_n| = |a_n - a_{m_0} + a_{m_0}| &\leq |a_n -a_{m_0}| + |a_{m_0}|<\epsilon_0 + |a_{m_0}|\\
\implies |a_n|&<\epsilon_0 + |a_{m_0}|
\end{aligned}\] ~~~~Thus, \(|a_n|\) is bounded.\\
\hspace*{0.333em}\hspace*{0.333em}\hspace*{0.333em}\hspace*{0.333em}Theorem
3.6.14 shows that a sequence is convergent if and only if it is Cauchy.
Since we know that Cauchy sequences are bounded, we can therefore deduce
that all convergent sequences are bounded.

\hfill \(\Box\)

\hypertarget{exercise-6}{%
\section{Exercise 6}\label{exercise-6}}

\textit{Let $(a_n)_{n \in \mathbb{N}}$ be a sequence of strictly positive real numbers and suppose that $(a_n)\to a$}

\begin{enumerate}
\item \textit{Show that $a \geq 0$.}  
Proof. Suppose $a < 0$. Note that $a_n > 0$ for all $n$. Then, since $a$ is the limit of $a_n$, for all $\epsilon > 0$, there exists $N$ such that for all $n > N$, $|a_n - a| < \epsilon$. Let $\epsilon = \frac{|a|}{2} > 0$. Since $a < 0$ and $a_n > 0$,
\[\begin{aligned}|a_n - a| = a_n - a &< \epsilon = -\frac{a}{2} \\
\implies a_n &< -\frac{a}{2} + a = \frac{a}{2}\end{aligned}\]
But $\frac{a}{2} < 0$, and $a_n > 0$, which is a contradiction. Thus, $a \geq 0$.  
  
\hfill $\Box$  
\item \textit{Show that $(\sqrt {a_n}) \to \sqrt{a}$}.  
Proof. Let $\epsilon = \frac{\epsilon'}{\sqrt a}$. Then, there exists $N \in \mathbb{N}$ such that for all $n > N$, the below holds.  
\[\begin{aligned} 
|\sqrt{a_n} - \sqrt a|
&= \frac{|\sqrt{a_n} - \sqrt a| \cdot |\sqrt{a_n} + \sqrt a|}{|\sqrt{a_n} + \sqrt a|} \\
&= \frac{|a_n - a|}{\sqrt{a_n} + \sqrt a} \\
&<\frac{|a_n - a|}{\sqrt a} \\
&< \frac{\epsilon}{\sqrt a} = \epsilon'
\end{aligned}\]
Thus, the limit of $\sqrt {a_n}$ is $\sqrt{a}$. 

\hfill $\Box$ 
\end{enumerate}

\hypertarget{exercise-7}{%
\section{Exercise 7}\label{exercise-7}}

\hypertarget{problem-3.6.18}{%
\subsection{Problem 3.6.18}\label{problem-3.6.18}}

\textit{Find the accumulation points of the following sets in $\mathbb{R}$}\\

\begin{enumerate}

\item $S = (0,1)$;  
  
Since the interval $(0,1)$ is continuous within its end points, it is clear that for any $x \in (0, 1)$, for all $\epsilon > 0$, $(x - \epsilon, x + \epsilon) \cap (0, 1) \backslash \{x\} \neq \emptyset$. Now, take $0$ and $1$. If $x = 0$, then there exists $x + \epsilon_0 \in (x - \epsilon, x + \epsilon)$ such that $x + \epsilon_0 \in (0, 1)$ where $\epsilon_0 < \epsilon, \epsilon_0 < 1$. Further, if $x = 1$, then there exists $x - \epsilon_0 \in (x - \epsilon, x + \epsilon)$ such that $x - \epsilon_0 \in (0, 1)$ where $\epsilon_0 < \epsilon, \epsilon_0 < 1$. Thus, $1$ and $0$ are included, and the accumulation points of $(0,1)$ are $[0, 1]$. 

\item \textit{$S = \{(-1)^n + \frac{1}{n}\}$};  
  
Let us observe the cases when $n$ is even and odd. When $n$ is even, we will denote this $n = 2m$ for $m \in \mathbb{N}$. Then, $(-1)^n + \frac{1}{n}$ gets arbitrarily close to $1$, since the greater $m$ is, the smaller $\frac{1}{n}$ gets, where $(-1)^n$ always equals $1$. In other words, for every $\epsilon > 0$, there exists $m \in \mathbb{N}$ such that $2m >  \frac{1}{\epsilon}$, or $\frac{1}{2m} < \epsilon$. Thus, $(1 - \epsilon, 1 + \epsilon) \cap S \backslash \{1\} \neq \emptyset$. So, $1$ is an accumulation point. Further, suppose $n$ is odd, and $n = 2m + 1$ for $m \in \mathbb{N}$. Then, $(-1)^n = -1$, and there exists $m \in \mathbb{N}$ such that $2m >  \frac{1}{\epsilon}$, or $\frac{1}{2m} < \epsilon$. Thus, $(-1 - \epsilon, -1 + \epsilon) \cap S \backslash \{-1\} \neq \emptyset$. So, $-1$ is an accumulation point.For any other $x \in \mathbb{R}$, let $s$ denote the closes number of the form $(-1)^n + \frac{1}{n}$. Then, for any $\epsilon < |x - s|$, there exists no number that can be represented as $(-1)^n + \frac{1}{n}$ within $(x - \epsilon, x + \epsilon)$, or else it would contradict our supposition that $s$ is the closest of such numbers to $x$. Thus, $x$ will not be an accumulation point, and the only accumulation points are $\{-1, 1\}$. 

\item $S = \mathbb{Q}$  ;
  
Take $x \in \mathbb{R}$. Suppose $x > 0$. Then, $x + \epsilon > 0$. Since $x, \epsilon \in \mathbb{R}$,  $\epsilon$ has a decimal expansion, either finite or infinite. Take this decimal expansion until the first nonzero digit in the decimal expansion of $\epsilon$ and denote this $\epsilon_1$. Then, $x < x + \epsilon_1 < x + \epsilon$. Now, take the decimal expansion of $x + \epsilon_1$ until the digit in the place after the place of the first digit of $\epsilon$. Denote this $(x + \epsilon_1)'$. Then, $x < (x + \epsilon_1)' < x + \epsilon$, and $(x + \epsilon_1)'$ is rational. Similarly, we can find that for any $x \leq 0$, $x - \epsilon < (x - \epsilon_1)' < x$. Thus, there exists a rational number in the neighborhood of every $x \in \mathbb{R}$, and the accumulation point of $\mathbb{Q}$ is $\mathbb{R}$. 

\item $S = \mathbb{Z}$;  
  
The integers do not have an accumulation point. Suppose they do. Then, for any $x \in \mathbb{R}$, take $s = \max\{|x - \lfloor{x}\rfloor| , |x - \lceil x \rceil|\}$ and let $\epsilon < s$. Then, for all $(x - \epsilon, x + \epsilon)$, there does not exist any integer. Thus, there is not accumulation point for $\mathbb{Z}$. 

\item \textit{S is the set of rational numbers whose denominators are prime.}  
  
We know that the sequence of primes is infinite and increasing, and therefore the sequence of fractions $\frac{1}{p_i}$, where $p_i$ is a prime, is infinite and decreasing. For every $\epsilon > 0$, there exists $p_i > \epsilon$. Thus, for every $\epsilon = \frac{1}{\epsilon}$, there is a $\frac{1}{p_i} < \epsilon$. Therefore, $(-\epsilon, \epsilon) \cap S \neq \emptyset \implies \{0\}$ is an accumulation point. For any other $x \in \mathbb{R}$, take $p_i$ such that $\frac{1}{p_i} < \epsilon$. Then, there exists at least one $\frac{k}{p_i} \in (x-\epsilon, x +\epsilon), k \in \mathbb{Z}$. Thus, all $x \in \mathbb{R}$ is an accumulation point of $S$.

\end{enumerate}

\hfill \(\Box\)

\hypertarget{exercise-8}{%
\section{Exercise 8}\label{exercise-8}}

\hypertarget{problem-3.6.21}{%
\subsection{Problem 3.6.21}\label{problem-3.6.21}}

\begin{enumerate}
\item \textit{Find an infinite subset of $\mathbb{R}$ that does not have an accumulation point in $\mathbb{R}$}  
  
$\mathbb{Z}$ is an infinite subset of $\mathbb{R}$ that does not have an accumulation point, as we saw earlier. 
\item \textit{Find a bounded subset of $\mathbb{R}$ that does not have an accumulation point in $\mathbb{R}$}  
  
The set $\{1\}$ is a bounded subset but does not have an accumulation point. This is because for any $x \in \mathbb{R}$, there exists a small enough $\epsilon > 0$ such that $(x - \epsilon, x + \epsilon) \backslash \{x\}$. 
\item \textit{Find a bounded infinite subset of $\mathbb{Q}$ that does not have an accumulation point in $\mathbb{Q}$}  
  
The sequence \[x_n = \lfloor \sqrt 2 \cdot  10 ^{n-1}\rfloor \cdot \frac{1}{10^{n-1}}\]
is bounded above by $\sqrt 2$, below by $1$, and is infinite. Its limit is $\sqrt 2$, as $x_n$ is just the decimal expansion of $\sqrt 2$ until the $n-1$th digit after the decimal. 
\end{enumerate}

\hfill \(\Box\)

\end{document}
