% Options for packages loaded elsewhere
\PassOptionsToPackage{unicode}{hyperref}
\PassOptionsToPackage{hyphens}{url}
%
\documentclass[
]{article}
\usepackage{lmodern}
\usepackage{amssymb,amsmath}
\usepackage{ifxetex,ifluatex}
\ifnum 0\ifxetex 1\fi\ifluatex 1\fi=0 % if pdftex
  \usepackage[T1]{fontenc}
  \usepackage[utf8]{inputenc}
  \usepackage{textcomp} % provide euro and other symbols
\else % if luatex or xetex
  \usepackage{unicode-math}
  \defaultfontfeatures{Scale=MatchLowercase}
  \defaultfontfeatures[\rmfamily]{Ligatures=TeX,Scale=1}
\fi
% Use upquote if available, for straight quotes in verbatim environments
\IfFileExists{upquote.sty}{\usepackage{upquote}}{}
\IfFileExists{microtype.sty}{% use microtype if available
  \usepackage[]{microtype}
  \UseMicrotypeSet[protrusion]{basicmath} % disable protrusion for tt fonts
}{}
\makeatletter
\@ifundefined{KOMAClassName}{% if non-KOMA class
  \IfFileExists{parskip.sty}{%
    \usepackage{parskip}
  }{% else
    \setlength{\parindent}{0pt}
    \setlength{\parskip}{6pt plus 2pt minus 1pt}}
}{% if KOMA class
  \KOMAoptions{parskip=half}}
\makeatother
\usepackage{xcolor}
\IfFileExists{xurl.sty}{\usepackage{xurl}}{} % add URL line breaks if available
\IfFileExists{bookmark.sty}{\usepackage{bookmark}}{\usepackage{hyperref}}
\hypersetup{
  pdftitle={Math 15910: Problem Set 6},
  pdfauthor={Underland, Jake},
  hidelinks,
  pdfcreator={LaTeX via pandoc}}
\urlstyle{same} % disable monospaced font for URLs
\usepackage[margin=1in]{geometry}
\usepackage{graphicx,grffile}
\makeatletter
\def\maxwidth{\ifdim\Gin@nat@width>\linewidth\linewidth\else\Gin@nat@width\fi}
\def\maxheight{\ifdim\Gin@nat@height>\textheight\textheight\else\Gin@nat@height\fi}
\makeatother
% Scale images if necessary, so that they will not overflow the page
% margins by default, and it is still possible to overwrite the defaults
% using explicit options in \includegraphics[width, height, ...]{}
\setkeys{Gin}{width=\maxwidth,height=\maxheight,keepaspectratio}
% Set default figure placement to htbp
\makeatletter
\def\fps@figure{htbp}
\makeatother
\setlength{\emergencystretch}{3em} % prevent overfull lines
\providecommand{\tightlist}{%
  \setlength{\itemsep}{0pt}\setlength{\parskip}{0pt}}
\setcounter{secnumdepth}{-\maxdimen} % remove section numbering

\title{Math 15910: Problem Set 6}
\author{Underland, Jake}
\date{2021-08-09}

\begin{document}
\maketitle

{
\setcounter{tocdepth}{2}
\tableofcontents
}
\hypertarget{exercise-1}{%
\section{Exercise 1}\label{exercise-1}}

\hypertarget{problem-1.5.4}{%
\subsection{Problem 1.5.4}\label{problem-1.5.4}}

\textit{Show that the statement "if $a,b \in \mathbb{Z}$ and $ab = 0$, then either $a = 0$ or $b=0$" is equivalent to cancellation. }

~~~~Proof. Using the fact that for all
\(a \in \mathbb{Z}, a \cdot 0 = 0\), we can restate the above as such:
\[ \text{ If } a \neq 0 \text{ and } ab = a \cdot 0, \text{ then } b = 0.\]
We replace the additive identity \(0\) with \(c\). Then, we get
\[ \text{ If } a \neq 0 \text{ and } ab = a \cdot c, \text{ then } b = c.\]
This is equivalent to the axiom of cancellation when \(b = 0\). When
\(b \neq 0\), by contrapositive of the above, since both
\(a, b \neq 0\), \(ab \neq 0\). Fix \(a, b, c \in \mathbb{Z}\) where
\(a, b, c \neq 0\). Suppose
\[\begin{aligned} ab &= cb \\ \implies ab - cb &= 0 \\ \implies (a - c)b &= 0 \\ \implies b = 0 &\text{ or } (a - c) = 0 \end{aligned}\]
Since \(b \neq 0\), we have that \(a - c = 0\) and thus \(a = c\).
Hence, we have arrived at the axiom of cancellation.

\hfill \(\Box\)

\hypertarget{exercise-2}{%
\section{Exercise 2}\label{exercise-2}}

\hypertarget{problem-1.5.7}{%
\subsection{Problem 1.5.7}\label{problem-1.5.7}}

\emph{Suppose that \(a < 0\) and \(b < 0\). Show that \(a < b\) iff
\(b^2 < a^2\)}

~~~~Proof. From Facts 1.5.5, We know that \(a < b \implies -a > -b\).
Since \(a, b < 0\) and \(-1= -1 + 0 < -1 + 1= 0\) (using fact \(1 >0\)
and (O3)), from fact 4, we have \(-a, -b > 0\). Thus, we can invoke
axiom (O4).
\[\begin{aligned} -a &> -b \\ \implies -a(-a) &> -b(-a) \\ \implies (-a)^2 &> -b(-a)\end{aligned} \]
Furthermore,
\[\begin{aligned} -a &> -b \\ \implies -a(-b) &> -b(-b) \\ \implies (-b)(-a) &> (-b)^2\end{aligned} \]
From (O2),
\[\begin{aligned} \implies (-a)^2 &> (-b)^2 \\ \implies (-1)(a)(-1)(a) &> (-1)(b)(-1)(b) \\ \implies (-1)^2 a^2 &> (-1)^2 b^2 \text{ (commutativity)} \end{aligned}\]
From fact 6 from facts 1.5.1, \(-1(-1) = 1\). Thus,
\[\implies a^2 > b^2\]\\
Now, suppose \(a^2 > b^2\). Then, reverse the steps above to derive
\(a < 0\) and \(b < 0\). \hfill \(\Box\)

\hypertarget{exercise-3}{%
\section{Exercise 3}\label{exercise-3}}

\hypertarget{problem-1.-5.-10}{%
\subsection{Problem 1. 5. 10}\label{problem-1.-5.-10}}

\emph{Show that the cancellation axiom can be proved using the
properties for addition and multiplication and the order axioms.}\\
\[\text{ If } a \neq 0 \text{ and } ab = ac, \text{ then } b = c.\]
~~~~Suppose \(a > 0\) and \(ab = ac\). Then, from (O4), if \(b < c\), it
would follow that \(ab < ac\). From (O1), this contradicts our
supposition, so \(b \geq c\). If \(b > c\), from (O4), it would follow
that \(ab > ac\). From (O1), this contradicts our supposition.
Therefore, \(b \leq c\). The only relation of \(b\) and \(c\) that
satisfies the two relations above is \(b = c\), from (O1).\\
\hspace*{0.333em}\hspace*{0.333em}\hspace*{0.333em}\hspace*{0.333em}Now,
suppose \(a < 0\) and \(ab = ac\). Then, from fact (7) from facts 1.5.5,
if \(b < c\), it would follow that \(ab > ac\). From (O1), this
contradicts our supposition. Therefore, \(b \geq c\). If \(b > c\), it
would follow that \(ab < ac\). From (O1), this contradicts our
assumption, so \(b\leq c\). From (O1), the only relation of \(b\) and
\(c\) that satisfies the above is \(b = c\).\\
\hspace*{0.333em}\hspace*{0.333em}\hspace*{0.333em}\hspace*{0.333em}Therefore,
when \(a \neq 0 \text{ and } ab = ac, \text{ then } b = c.\)

\hfill \(\Box\)

\hypertarget{exercise-4}{%
\section{Exercise 4}\label{exercise-4}}

\hypertarget{problem-1.6.21}{%
\subsection{Problem 1.6.21}\label{problem-1.6.21}}

\emph{Let \(X\) be a nonempty set and \(R = P(X)\). Show that \(R\) with
symmetric difference as addition and intersection as multiplication is a
commutative ring with \(1\). When is \(R\) a field?}

\begin{itemize} 
\item (A1): We define addition as the symmetric difference of two sets. Since $P(X)$ is the set of all possible subsets of $X$, any nonempty set whose elements are also in $X$ is in $P(X)$, as well as the empty set. It is clear that the symmetric difference of any two elements in $P(X)$, which are both subsets of $X$, will be a subset of $X$ and thus contained in $P(X)$.  
Take $A, B \in R$. Suppose $A \triangle B \notin R$. Then, there exists $z \in A \triangle B$ such that $z \notin X$. By definition of symmetric difference, $z$ must be an element of either $A$ or $B$. However, if $z$ was in either set, then that set would not be a subset of $X$, and thus not in $R$. This is a contradiction, so $A \triangle B \in R$, and we have shown closure of addition in $R$. 
\item (A2): From facts presented in Sally 1.3, we know that symmetric difference is associative. 
\item (A3): Similarly, we know that symmetric difference is commutative. 
\item (A4): Take any set $A \in R$. Then, \[\begin{aligned} &A \triangle \emptyset \\ = &\emptyset \triangle A \space \dots \text{ from associativity established above}\\ = &(A\backslash \emptyset) \cup (\emptyset \backslash A) \\ = & A \cup \emptyset \\ =& A \end{aligned} \]
Thus, the additive identity exists and is the empty set. 
\item (A5): Take $A \in R$. Then, 
\[\begin{aligned} & A \triangle A \\ = &(A\backslash A) \cup (A \backslash A) \\ = & \emptyset \cup \emptyset \\ =& \emptyset \end{aligned}\]
Thus, the additive inverse for each set exists and is the set itself.
  

\item (M1): We define multiplication as the intersection of two sets.  
Take $A, B \in R$. Suppose $A \cap B \notin R$. Then, there exists $z \in A\cap B$ such that $z \notin X$. By definition of intersection, $z$ must be an element of both $A$ and $B$. However, if $z$ was in either set, then that set would not be a subset of $X$, and thus not in $R$. This is a contradiction, so $A \cap B \in R$, and we have shown closure of multiplication in $R$. 
\item (M2): From facts presented in Sally 1.3, we know that intersection is associative.  
\item (M3): Similarly, we know that intersection is commutative. 
\item (M4): Take any set $A \in R$, and the set $B \in R$ such that $B = X$. Then, \[\begin{aligned} &A \cap B \\ = &B \cap A \space \dots \text{ from associativity established above}\\ = &\{a \in A| a \in X\}\end{aligned}\] 
Since $A \subseteq X$, 
\[\begin{aligned} &\{a \in A| a \in X\} \\ =& \{a \in A\} \\ = & A\end{aligned} \]
Thus, the multiplicative identity for each set exists and is the set $X$.   


\item (D): From the distributive law of intersection over symmetric difference shown in 1.3 of Sally, intersection is distributive over symmetric difference. 
\end{itemize}

Therefore, \(R = P(X)\) satisfies the axioms required for it to be a
commutative ring with \(1\). \(R\) is a field when cancellation holds
for multiplication and it also satisfies (M5). Since the multiplicative
identity in this set is \(X\), for there to be a multiplicative inverse
for any set in \(R\), then the intersection of said set with the inverse
must equal \(X\). This means the intersection of the emptyset with the
inverse, which will inevitably be the empty set, must equal \(X\), and
so \(X = \emptyset\). Then, cancellation of multiplication, or
\(\emptyset \cap \emptyset = \emptyset \cap \emptyset \implies \emptyset = \emptyset\)
will also hold since there is only one element in \(R\), and thus it
must be equal to itself.

\hfill \(\Box\)

\hypertarget{exercise-5}{%
\section{Exercise 5}\label{exercise-5}}

\hypertarget{problem-3.1.5}{%
\subsection{Problem 3.1.5}\label{problem-3.1.5}}

\emph{Show that the least upper bound of a set is unique, if it exists}

~~~~Proof. Let \(F\) be an ordered field, and \(A\) a non-empty subset
of \(F\) that is bounded above. Suppose that the least upper bound of
\(A\) exists. We assume \(L,M\) are both the least upper bound of \(A\).
Then, \(L, M\) are both upper bounds for \(A\). Furthermore, since \(M\)
is an upper bound and \(L\) is a least upper bound, by definition,
\(L \leq M\). Similarly, since \(L\) is an upper bound and \(M\) is a
least upper bound, \(M \leq L\). From (O1), this would imply that
\(L = M\). Thus, the least upper bound of a set is unique if it exists.

\hfill \(\Box\)

\hypertarget{exercise-6}{%
\section{Exercise 6}\label{exercise-6}}

\hypertarget{problem-3.1.11}{%
\subsection{Problem 3.1.11}\label{problem-3.1.11}}

\emph{Prove that an ordered field has the least upper bound property iff
it has the greatest lower bound property}

~~~~Suppose \(F\) is an ordered field with the least upper bound
property. Then, take any nonempty \(A \subseteq F\) that is bounded
below. Because \(A\) is bounded below, it has a set of lower bounds. Let
\(B\) be this set. Then, since \(B\) is the set of lower bounds for
\(A\), by definition, any \(a \in A\) is an upper bound of \(B\). Thus,
\(B\) is bounded above. Because the field has a least upper bound
property, and \(B\) is a subset of said field bounded above, \(B\) has a
least upper bound. Since \(B\) is a set of lower bounds of \(A\),
\(\sup B\) is a lower bound for \(A\). By definition of supremum, there
is no lower bound for \(A\) that is greater than \(\sup B\), for if
there was, it would be an element in \(B\) that is greater than
\(\sup B\), and so \(\sup B\) would fail to be an upper bound for \(B\).
To paraphrase, if \(m\) is any lower bound for \(A\), \(m \leq \sup B\).
Thus, \(\sup B = \inf B\), and any non-empty subset \(A\) of \(F\) has a
greatest lower bound, giving \(F\) the greatest lower bound property.\\
\hspace*{0.333em}\hspace*{0.333em}\hspace*{0.333em}\hspace*{0.333em}With
analogous reasoning, we can formulate a proof for any \(F\) with the
greatest lower bound property having the least upper bound property. It
follows that an ordered field has the least upper bound property if and
only if it has the greatest lower bound property.

\hfill \(\Box\)

\hypertarget{exercise-7}{%
\section{Exercise 7}\label{exercise-7}}

\hypertarget{problem-3.1.14}{%
\subsection{Problem 3.1.14}\label{problem-3.1.14}}

\emph{Suppose that \(A\) and \(B\) are bounded sets in \(\mathbb{R}\).
Prove or disprove the following:}\\

\begin{enumerate} 
\item $\sup (A \cup B) = \max\{\sup (A), \sup (B)\}$  

Proof. \linebreak  
If $\sup A > \sup B$, then $\sup B$ cannot be $\sup(A \cup B)$, because there exists $a \in A$ such that $a > \sup B$, and therefore $a \in A \cup B$ such that $a > \sup B$, and $\sup B$ would not be an upper bound of $A \cup B$. If $\sup B$ was an upper bound of $A \cup B$ and there was no $a \in A$ such that $a > \sup B$, then $\sup B$ would be an upper bound of $A$ and $\sup B < \sup A$, contradicting the fact that $\sup A$ is the least upper bound of $A$. Thus, we know that if $\sup A > \sup B$, $\sup B$ is not a least upper bound of $A \cup B$. In this case, since $\sup A > \sup B$ and for all $b \in B$, $b < \sup B$, then $b < \sup A$. This is also true for all $a \in A$, and so $\sup A$ is an upper bound of $A \cup B$. Since $\sup A$ is the least upper bound of $A$, any upper bound of $A$ is greater than $\sup A$. For any upper bound of $B$, if the upper bound is less than $A$, there exists $a \in A$ such that $a$ is greater than the upper bound, meaning it is not an upper bound for $A \cup B$. Thus, $\sup (A \cup B) = \sup A$. \linebreak
Similarly, if $\sup A < \sup B$, then $\sup (A \cup B) = \sup B$. Hence, $\sup (A \cup B) = \max\{\sup (A), \sup (B)\}$.  

\hfill $\Box$
\item \textit{If $A+B = \{a + b | a \in A, b \in B\}$, then $\sup (A + B) = \sup (A) + \sup (B)$}  

Proof. \linebreak
For all $a \in A$ and $b \in B$, $a \leq \sup A$ and $b \leq \sup B$, implying $a + b \leq \sup A + \sup B$. Thus, $\sup A + \sup B$ is an upper bound of $A + B$.  

Now, take $a_ \epsilon \in A$, $b_\epsilon \in B$ such that $a_\epsilon > \sup A - \frac{\epsilon}{2}$, and $b_\epsilon > \sup B - \frac{\epsilon}{2}$ for any $\epsilon > 0$. We know these exist from the definition of supremum. Then, there would exist $a_\epsilon + b_\epsilon > \sup A + \sup B - \epsilon$, and $\sup A + \sup B - \epsilon$ is not an upper bound of $A+B$. Thus, the least upper bound is $\sup A + \sup B$.  
\hfill $\Box$
\item \textit{If the elements of $A$ and $B$ are positive and $A \cdot B = \{ab | a \in A, b \in B\}$, then $\sup (A \cdot B) = \sup A \sup B$.}  

Proof.  

For all $a \in A$ and $b \in B$, $0 < a \leq \sup A$ and $0 < b \leq \sup B$, implying $ab \leq \sup A  \sup B$. Thus, $\sup A \sup B$ is an upper bound of $A \cdot B$.  

Furthermore, take $a_ \epsilon \in A$, $b_\epsilon \in B$ such that $a_\epsilon > \sup A - \sqrt{\epsilon} > 0$, and $b_\epsilon > \sup B - \sqrt{\epsilon} > 0$ for any $\epsilon > 0$. We set $\sup A - \sqrt{\epsilon} > 0$ and $\sup B - \sqrt{\epsilon} > 0$ because the set of $A$ and $B$ are positive real numbers, and for these values to be contained in or be an upper bound of $A$ or $B$, we need them to be positive. We know these exist from the definition of supremum. Then, there would exist $a_\epsilon \cdot b_\epsilon \in A \cdot B$ such that
\[\begin{aligned} a_\epsilon \cdot b_\epsilon &> \sup A \cdot \sup B - \sqrt{\epsilon}(\sup A + \sup B) + \epsilon \\ \implies a_\epsilon \cdot b_\epsilon &> \sup A \cdot \sup B - \sqrt{\epsilon}[\sup A + \sup B - \sqrt{\epsilon}] \end{aligned}\]
and $\sup A \cdot \sup B - \sqrt{\epsilon}[\sup A + \sup B - \sqrt{\epsilon}]$ is not an upper bound of $A \cdot B$. 
Since $\sup A > 0$ and $\sup B - \sqrt{\epsilon} > 0$ and $\epsilon > 0$, we know that $\sqrt{\epsilon}[\sup A + \sup B - \sqrt{\epsilon}] > 0$. We can restate this as $\epsilon '$, and conclude that the least upper bound of $A \cdot B$ is $\sup A \cdot \sup B$.  
\hfill $\Box$

\item \textit{Formulate the analogous problems for the greatest lower bound.}
\begin{enumerate} 
\item $\inf (A \cup B) = \min \{\inf A, \inf B\}$  
\item If $A+B = \{a + b | a \in A, b \in B\}$, then $\inf (A + B) = \inf (A) + \inf (B)$
\item If the elements of $A$ and $B$ are positive and $A \cdot B = \{ab | a \in A, b \in B\}$, then $\inf (A \cdot B) = \inf A \inf B$.
\end{enumerate}
\end{enumerate}

\hypertarget{exercise-8}{%
\section{Exercise 8}\label{exercise-8}}

\emph{Suppose that set \(S \subseteq \mathbb{R}\) contains one of its
upper bounds. Show that this upper bound is the supremum of \(S\)}

~~~~We denote this upper bound to be \(L\). It is given that \(L\) is an
upper bound of \(S\).\\
\hspace*{0.333em}\hspace*{0.333em}\hspace*{0.333em}\hspace*{0.333em}Now,
for all \(\epsilon > 0\), take \(L - \epsilon\). Then,
\(L > L - \epsilon\) and \(L \in S\), so \(L - \epsilon\) cannot be an
upper bound of \(S\). Any upper bound of \(S\) must be at least as great
as \(L\). We already have that \(L\) is an upper bound, so from here we
can conclude that \(L\) is the least upper bound of \(S\).
\hfill \(\Box\)

\end{document}
