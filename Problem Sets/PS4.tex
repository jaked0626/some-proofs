% Options for packages loaded elsewhere
\PassOptionsToPackage{unicode}{hyperref}
\PassOptionsToPackage{hyphens}{url}
%
\documentclass[
]{article}
\usepackage{lmodern}
\usepackage{amssymb,amsmath}
\usepackage{ifxetex,ifluatex}
\ifnum 0\ifxetex 1\fi\ifluatex 1\fi=0 % if pdftex
  \usepackage[T1]{fontenc}
  \usepackage[utf8]{inputenc}
  \usepackage{textcomp} % provide euro and other symbols
\else % if luatex or xetex
  \usepackage{unicode-math}
  \defaultfontfeatures{Scale=MatchLowercase}
  \defaultfontfeatures[\rmfamily]{Ligatures=TeX,Scale=1}
\fi
% Use upquote if available, for straight quotes in verbatim environments
\IfFileExists{upquote.sty}{\usepackage{upquote}}{}
\IfFileExists{microtype.sty}{% use microtype if available
  \usepackage[]{microtype}
  \UseMicrotypeSet[protrusion]{basicmath} % disable protrusion for tt fonts
}{}
\makeatletter
\@ifundefined{KOMAClassName}{% if non-KOMA class
  \IfFileExists{parskip.sty}{%
    \usepackage{parskip}
  }{% else
    \setlength{\parindent}{0pt}
    \setlength{\parskip}{6pt plus 2pt minus 1pt}}
}{% if KOMA class
  \KOMAoptions{parskip=half}}
\makeatother
\usepackage{xcolor}
\IfFileExists{xurl.sty}{\usepackage{xurl}}{} % add URL line breaks if available
\IfFileExists{bookmark.sty}{\usepackage{bookmark}}{\usepackage{hyperref}}
\hypersetup{
  pdftitle={Math 15910: Problem Set 4},
  pdfauthor={Underland, Jake},
  hidelinks,
  pdfcreator={LaTeX via pandoc}}
\urlstyle{same} % disable monospaced font for URLs
\usepackage[margin=1in]{geometry}
\usepackage{graphicx,grffile}
\makeatletter
\def\maxwidth{\ifdim\Gin@nat@width>\linewidth\linewidth\else\Gin@nat@width\fi}
\def\maxheight{\ifdim\Gin@nat@height>\textheight\textheight\else\Gin@nat@height\fi}
\makeatother
% Scale images if necessary, so that they will not overflow the page
% margins by default, and it is still possible to overwrite the defaults
% using explicit options in \includegraphics[width, height, ...]{}
\setkeys{Gin}{width=\maxwidth,height=\maxheight,keepaspectratio}
% Set default figure placement to htbp
\makeatletter
\def\fps@figure{htbp}
\makeatother
\setlength{\emergencystretch}{3em} % prevent overfull lines
\providecommand{\tightlist}{%
  \setlength{\itemsep}{0pt}\setlength{\parskip}{0pt}}
\setcounter{secnumdepth}{-\maxdimen} % remove section numbering

\title{Math 15910: Problem Set 4}
\author{Underland, Jake}
\date{2021-08-09}

\begin{document}
\maketitle

{
\setcounter{tocdepth}{2}
\tableofcontents
}
\hypertarget{exercise-1}{%
\section{Exercise 1}\label{exercise-1}}

\emph{Show that if \(A\) and \(B\) are countably infinite sets, then
there exists a bijection between their power sets \(P(A)\) and
\(P(B)\).}

~~~~Since \(A\) and \(B\) are countably infinite, there exist bijective
functions \(f:A \to \mathbb{N}\), \(g:B \to \mathbb{N}\). Then,
\(g^{-1} \circ f\) is a bijective function between \(A\) and \(B\).
Thus, we know that \(A\) and \(B\) are bijective. Let us denote
\(h: A \to B\) to be an arbitrary bijection. Now think of the function
\(q: P(A) \to P(B)\) which takes any element of \(P(A)\),
\(\{a_1, a_2, ... a_n\} \subseteq A\), and returns
\(\{h(a_1), h(a_2), ... , h(a_n)\} = \{b_1, b_2, ..., b_n\} \subseteq B\).

~~~~Take \(\{a_1, ..., a_n\}, \{a'_1, ..., a'_n\} \in P(A)\).\\
If,\\
\[\begin{aligned} q(\{a_1, a_2, ..., a_n\}) &=  q(\{a'_1, a'_2,..., a'_n\}) \\ \implies \{h(a_1), h(a_2), ..., h(a_n)\}) &=  \{h(a'_1), h(a'_2),..., h(a'_n)\} \end{aligned}\]

From the definition of equality of sets, and without loss of
generality,\\
\[h(a_1) = h(a'_1), h(a_2) = h(a'_2) ..., h(a_n) = h(a'_n)\] Since \(h\)
is bijective,\\
\[ \begin{aligned} \implies \{h^{-1}(h(a_1)), h^{-1}(h(a_2)), ..., h^{-1}(h(a_n))\}) &=  \{h^{-1}(h(a'_1)), h^{-1}(h(a'_2)),..., h^{-1}(h(a'_n))\} \\ \implies \{a_1, a_2, ..., a_n\} &=  \{a'_1, a'_2,..., a'_n\} \end{aligned} \]
From the definition of equality of sets, and without loss of
generality,\\
\[a_1 = a'_1, a_2 = a'_2 ..., a_n = a'_n\] Thus, \(q\) is injective.

~~~~Now, take \(\{b_1, \dots, b_n \} \in P(B)\). We want
\(\{a_1, \dots, a_n \}\in P(A)\) such that
\(q(\{a_1, \dots, a_n \}) = \{b_1, \dots, b_n \}\).\\
\[ \begin{aligned} q(\{a_1, ..., a_n\}) &= \{b_1, \dots, b_n \} \\
\implies \{h(a_1), ..., h(a_n)\}) &= \{b_1, \dots, b_n \} \\ \implies \{a_1, \dots, a_n \} &= \{h^{-1}(b_1), \dots, h^{-1}(b_n) \} \end{aligned} \]
Thus, given \(\{b_1, \dots, b_n \} \in P(B)\),
\(q(\{h^{-1}(b_1), \dots, h^{-1}(b_n) \}) = \{b_1, \dots, b_n \}\).
Hence, \(q\) is surjective.\\
\hspace*{0.333em}\hspace*{0.333em}\hspace*{0.333em}\hspace*{0.333em}It
follows that \(q\) is bijective. With this, we have shown that there
exists a bijection between \(P(A)\) and \(P(B)\) given two countably
infinite sets \(A\) and \(B\).

\hfill \(\Box\)

\hypertarget{exercise-2}{%
\section{Exercise 2}\label{exercise-2}}

\hypertarget{problem-1.8.31}{%
\subsection{Problem 1.8.31}\label{problem-1.8.31}}

\emph{Suppose that \(A\) is a non-empty set. Show that \(P(A)\) is in
one to one correspondence with the set of all functions from \(A\) to
\(\{0, 1\}\)}

~~~~For every subset \(X \subseteq A\), we define a function
\(f_X : X \to \{0,1\}\) such that for every \(a \in A\),
\[f_X(a) = \begin{cases} 1, \text{    if } a\in X \\ 0, \text{    if } a \notin X \end{cases}\]
Now, we define the function \(F: P(A) \to \{f|f:A \to \{ 0,1\}\}\) that
takes a subset \(B \in P(A)\) and returns \(f_B\) from the set of
functions \(\{f|f:A \to \{ 0,1\}\}\).\\
\hspace*{0.333em}\hspace*{0.333em}\hspace*{0.333em}\hspace*{0.333em}Suppose
\(B, C \subseteq A\) and \(F(B) = F(C)\). Then,
\[ \begin{aligned} f_B &= f_C \\
\implies f_B(a) &= f_C(a) \space \space \space \text{    for every } a \in A\end{aligned}\]
Since \(B, C \subseteq A\), there does not exist any element within them
that is not also in \(A\). If \(f_B(a) = f_C(a) = 1\), then
\(a \in B, C\). If \(f_B(a) = f_C(a) = 0\), then \(a \notin B, C\). By
the definition of set equality, \(B = C\). Thus, \(F\) is injective.\\
\hspace*{0.333em}\hspace*{0.333em}\hspace*{0.333em}\hspace*{0.333em}Now,
for every \(f_X \in \{f| f:A \to \{ 0,1\}\}\) there exists
\(X \subseteq A\) such that \(F(X) = f_X\). \[F(X) = f_X\]
\[\implies X = \{a \in A| f_X(a) = 1\}\] Thus, we have that for every
\(f_X \in \{f| f:A \to \{ 0,1\}\}\), \(F(\{a\in A|f_X(a)=1\}) = f_X\),
and \(F\) is surjective.\\
\hspace*{0.333em}\hspace*{0.333em}\hspace*{0.333em}\hspace*{0.333em}It
follows that \(F\) is bijective, and \(P(A)\) is in one to one
correspondence with the set of all functions from \(A\) to \(\{0, 1\}\).

\hfill \(\Box\)

\hypertarget{exercise-3}{%
\section{Exercise 3}\label{exercise-3}}

\emph{Prove that a union \(\cup _{n \in \mathbb{N}} A_n\) where sets
\(A_n\) have the cardinality of the set of real numbers, has the
cardinality of the set of real numbers.}

~~~~We know that there is a bijection between \((0, 1)\) and
\(\mathbb{R}\), so \((0, 1)\) has the same cardinality as
\(\mathbb{R}\). In the last homework, I showed that there exists a
bijection \(f: [0, 1) \to (0, 1)\), defined as
\[f(x) = \begin{cases} \frac{n+1}{n+2} \space \space \text{  if } x = \frac{n}{n+1} \text{ for } n \in \mathbb{N}_0 \\ x \space \space\text{ for all other } x \in [0,1) \end{cases}\]
Thus, \([0, 1)\) has the same cardinality as the set of real numbers.
Similarly, we can define functions for \([0,1), [1, 2), ..., [n-1, n)\)
that take the form
\[f_n(x) = \begin{cases} \frac{m+1}{m+2} \space \space \text{  if } x = n - 1 +\frac{m}{m+1} \text{ for } m \in \mathbb{N}_0 \\ x - n + 1 \space \space\text{ for all other } x \in [n -1,n) \end{cases}\]
which are all bijections with \((0,1)\), and thus have equal cardinality
to the set of real numbers. Since each \(A_n\) has the cardinality of
the set of real numbers, and the sets \([0,1), [1, 2), ...,[n-1, n)\)
also have the same cardinality, there exists a bijection between \(A_n\)
and \([n-1, n)\). We now show that there is an injection from
\(\cup _{n \in \mathbb{N}}A_n\) to \([0, n)\) and an injection from
\([0, n)\) to \(A_n\).\\
\hspace*{0.333em}\hspace*{0.333em}\hspace*{0.333em}\hspace*{0.333em}Since
for all \(A_i\) for \(1 \leq i \leq n\), \(A_i\) is bijective with
\([i - 1, i)\), there exists an injection in this direction. If all
\(A_i\) are mutually disjoint, they would each have an injection from
\(A_i\) to \([i - 1, i)\), and thus the union
\(\cup _{n \in \mathbb{N}}A_n\) will inject to \([0, n)\), which is the
union of all \([i, i-1)\). If the sets are not mutually disjoint, then
each \(a \in \cup _{n \in \mathbb{N}}A_n\) will inject to the
\([i, i-1)\) corresponding to the first \(A_i\) in which \(a\) appears.
It is clear that this is an injection, and thus there is an injection
from \(\cup _{n \in \mathbb{N}}A_n\) to \([0, n)\).\\
\hspace*{0.333em}\hspace*{0.333em}\hspace*{0.333em}\hspace*{0.333em}Next,
since \([0, 1)\) is bijective to \(A_1\), there exists an interjection
from \([0, 1)\) to \(A_1\). Then, by multiplying all elements in
\([0, 1)\) by \(n\), we can create an injection from \([0, n)\) to
\(A_1\). Since \(A_1 \in \cup _{n \in \mathbb{N}}A_n\), it follows that
there exists an injection from \([0, n)\) to \(A_n\).\\
\hspace*{0.333em}\hspace*{0.333em}\hspace*{0.333em}\hspace*{0.333em}Since
there is an injection from \(\cup _{n \in \mathbb{N}}A_n\) to \([0, n)\)
and an injection from \([0, n)\) to \(A_n\), we can invoke the
Schroeder-Bernstein theorem to conclude that there exists a bijection
between \(\cup _{n \in \mathbb{N}}A_n\) and \([0, n)\). It is easy to
see that there exists a bijective function \(f: [0, 1) \to [0, n)\) such
that \[f(x) = nx\] It is trivial to prove that this function is
bijective. Thus, we know that {[}0, n) has the same cardinality as
\(\mathbb{R}\).\\
\hspace*{0.333em}\hspace*{0.333em}\hspace*{0.333em}\hspace*{0.333em}From
here, we can conclude that a union \(\cup _{n \in \mathbb{N}} A_n\)
where sets \(A_n\) have the cardinality of the set of real numbers, has
the cardinality of the set of real numbers.

\hfill \(\Box\)

\hypertarget{exercise-4}{%
\section{Exercise 4}\label{exercise-4}}

\emph{Prove that the set of irrational numbers has the same cardinality
as the set of real numbers.}

~~~~Consider the function
\[f(x) = \begin{cases} \arctan{x} \space \space \space \text{ when } \arctan{x} \in \mathbb{R} \backslash \mathbb{Q} \\ \arctan{x + 10 \sqrt{2}} \space \space \space \text{ when } \arctan{x} \in \mathbb{Q} \end{cases}\]
Given \(x \in \mathbb{R}\), this function produces an irrational number,
so this function is well defined.\\
\hspace*{0.333em}\hspace*{0.333em}\hspace*{0.333em}\hspace*{0.333em}Now,
take \(x_1, x_2 \in \mathbb{R}\). Suppose \(f(x_1) = f(x_2)\). Then,
either \(\arctan{x_1} \text{ and } \arctan{x_2}\) are both rational or
both irrational. If \(x_1\) was rational and \(x_2\)was irrational, then
\(\arctan{x_1}+10\sqrt{2} \neq \arctan{x_2}\) since the maximum possible
value of \(\arctan x\) is \(\frac{\pi}{2}\) and the minimum is
\(-\frac{\pi}{2}\), making \(\arctan{x_1}+10\sqrt{2} > \arctan{x_2}\).\\
\hspace*{0.333em}\hspace*{0.333em}\hspace*{0.333em}\hspace*{0.333em}When
\(\arctan{x_1} \text{ and } \arctan{x_2}\) are both rational,
\[f(x_1) = f(x_2)\]
\[\implies \arctan{x_1}+10\sqrt{2} = \arctan{x_2}+10\sqrt{2}\]
\[\implies \arctan{x_1} = \arctan{x_2}\]
\[\implies \tan(\arctan{x_1}) = \tan(\arctan{x_2})\]
\[\implies x_1 = x_2\] ~~~~When
\(\arctan{x_1} \text{ and } \arctan{x_2}\) are both rational,
\[ \begin{aligned} f(x_1) &= f(x_2) \\
\implies \arctan{x_1} &= \arctan{x_2} \\
\implies \tan(\arctan{x_1}) &= \tan(\arctan{x_2}) \\
\implies x_1 &= x_2 \end{aligned} \] ~~~~Thus, \(f\) is an interjection.
Since \(\mathbb{R} \backslash \mathbb{Q} \subseteq \mathbb{R}\), it is
trivial to prove that there exists an injection
\(g: \mathbb{R} \backslash \mathbb{Q} \to \mathbb{R}\) (the identity
function would be such an injection).\\
\hspace*{0.333em}\hspace*{0.333em}\hspace*{0.333em}\hspace*{0.333em}Hence,
we can invoke the Schroeder-Bernstein theorem to conclude that there
exists a bijection between the set of irrational numbers and the set of
real numbers, and that they have the same cardinality.

\hfill \(\Box\)

\hypertarget{exercise-5}{%
\section{Exercise 5}\label{exercise-5}}

\hypertarget{problem-1.6.2}{%
\subsection{Problem 1.6.2}\label{problem-1.6.2}}

\emph{Let \(\mathcal{R}\) be a relation on \(X\) that satisfies}\\
\hspace*{0.333em}\hspace*{0.333em}\hspace*{0.333em}\hspace*{0.333em}(a.)
\emph{For all \(a \in X, (a, a) \in \mathcal{R}\), and}\\
\hspace*{0.333em}\hspace*{0.333em}\hspace*{0.333em}\hspace*{0.333em}(b.)
\emph{for
\(a, b, c \in X, \text{ if } (a, b), (b, c) \in \mathcal{R}, \text{ then } (c, a) \in \mathcal{R}\)}

~~~~From the assumption (a.), we already have reflexivity. Now, take
\((a, b) \in \mathcal{R}\). From (a.), we have
\((b, b) \in \mathcal{R}\). From (b.), if \((a, b) \in \mathcal{R}\), we
can say \[(a, b), (b, b) \in \mathcal{R}\]
\[\implies (b, a) \in \mathcal{R}\] Thus, we obtain symmetry. With
symmetry, it can be easily shown from (b.) that
\[(a, b), (b, c) \in \mathcal{R} \implies (c, a) \in \mathcal{R}\]
\[(c, a) \in \mathcal{R} \implies (a, c) \in \mathcal{R}\] Thus,
\[(a, b), (b, c) \in \mathcal{R} \implies (a,c ) \in \mathcal{R}\]
~~~~Therefore, we have established reflexivity, symmetry, and
transitivity in \(\mathcal{R}\), and have proven that \(\mathcal{R}\) is
an equivalence relation.

\hfill \(\Box\)

\hypertarget{exercise-6}{%
\section{Exercise 6}\label{exercise-6}}

\hypertarget{problem-1.6.14}{%
\subsection{Problem 1.6.14}\label{problem-1.6.14}}

\emph{Take a set \(X\) and break it up into pairwise disjoint non-empty
subsets whose union is all of \(X\). Then, for \(a, b \in X\), define
\(a \sim b\) if \(a\) and \(b\) are in the same subset. Prove that this
is an equivalence relation}

~~~~We denote these subsets as \(Y_i\). Suppose \(a \in Y_a\). Then,
since \(a \in Y_a\), \(a \sim a\) and we have reflexivity. If
\(a \sim b\), then \(a, b \in Y_a\). However, this would also imply
\(b \sim a\). Thus, we have symmetry. If \(a \sim b\), then
\(a, b \in Y_a\). Furthermore, if \(b \sim c\), then \(b, c \in Y_b\).
However, since \(Y_i\)'s are pairwise disjoint, it follows from
\(b \in Y_a, b \in Y_b\) that \(Y_a = Y_b\). thus, \(a, c \in Y_a\) and
\(a \sim c\). From here, we have transitivity. Thus, the above is an
equivalence relation.

\hfill \(\Box\)

\hypertarget{exercise-7}{%
\section{Exercise 7}\label{exercise-7}}

\emph{Let \(A\) be a set, and \(P(A)\) its power set. For
\(x,y\in P(A)\) let \(x \sim y\) if \(x\) and \(y\) have the same
cardinality. Prove that \(\sim\) is an equivalence relation.}\\
\emph{Compute the equivalence classes when \(A = \{1, 2, 3\}\).}

~~~~Take \(x \in P(A)\). Then, since \(|x| = |x|\), \(x \sim x\). Thus,
\(\sim\) has reflexivity. For \(x, y \in P(A)\), if \(x \sim y\), then
\(|x| = |y|.\) If \(y \sim x\), then \(|y| = |x|\). Thus,
\(x \sim y \implies y \sim x\). Hence, we have symmetry. If
\(x \sim y\), then \(|x| = |y|\). Furthermore, if \(y \sim z\), then
\(|y| = |z|\). It follows that \(|x| = |z|\), and \(x \sim z\). From
here, we have transitivity. Thus, the above is an equivalence
relation.\\
\hspace*{0.333em}\hspace*{0.333em}\hspace*{0.333em}\hspace*{0.333em}The
equivalence classes for when \(A = \{1, 2, 3\}\) are as follows:\\

\begin{itemize}
\item $\{x \in P(A)| |x| = 0\} = \{\emptyset\}$  
\item $\{x \in P(A)| |x| = 1\} = \{\{1\}, \{2\}, \{3\}\}$  
\item $\{x \in P(A)| |x| = 2\} = \{\{1, 2\}, \{1, 3\}, \{2, 3\}\}$  
\item $\{x \in P(A)| |x| = 3\} = \{\{1, 2, 3\}\}$  
\end{itemize}

\hfill \(\Box\)

\hypertarget{exercise-8}{%
\section{Exercise 8}\label{exercise-8}}

\hypertarget{problem-1.6.15}{%
\subsection{Problem 1.6.15}\label{problem-1.6.15}}

\emph{We consider the set
\(F = \{\{a, b\}|a, b\in \mathbb{Z} \text{ and }b \neq 0\}\). For
\((a,b),(c,d) \in F\), we define \((a,b) \sim (c,d)\) if \(ad = bc\).
Thus, for instance, \((2,3) \sim (8, 12) \sim (-6, -9)\).}\\
\emph{Show that \(\sim\) is an equivalence relation on \(F\)}

~~~~Take \((x, y) \in F\). Then, it is obvious that \(xy = xy\), so
\((x, y) \sim (x, y)\). So, \(\sim\) is reflexive. Now, take
\((x_1, y_1), (x_2, y_2) \in F\).
\[(x_1, y_1) \sim (x_2, y_2) \implies x_1y_2 = x_2y_1\] Since equality
\(=\) is an equivalence relation,
\[\implies x_2y_1=x_1y_2 \implies (x_2, y_2) \sim (x_1, y_1)\] Hence, we
have symmetry. Assume \((x, y) \sim (q, r)\) and hence \(xr = yq\).
Furthermore, assume \((q,r) \sim (s,t)\) and hence \(qt = rs\). Then,
\[ \begin{aligned} xr &= yq \\
\implies xr * t &= yq * t \\
\implies xrt &= yqt \end{aligned} \] From \(qt = rs\),
\[\implies xrt = yrs\] Since \(r \neq 0\), \[\implies xt = ys\]
\[\implies (x, y) \sim (s, t)\] From here, we have
\((x, y) \sim (q, r), (q, r) \sim (s, t) \implies (x, y) \sim (s, t)\).
Thus, \(\sim\) is transitive, and is an equivalence relation.

\hfill \(\Box\)

\end{document}
