% Options for packages loaded elsewhere
\PassOptionsToPackage{unicode}{hyperref}
\PassOptionsToPackage{hyphens}{url}
%
\documentclass[
]{article}
\usepackage{lmodern}
\usepackage{amssymb,amsmath}
\usepackage{ifxetex,ifluatex}
\ifnum 0\ifxetex 1\fi\ifluatex 1\fi=0 % if pdftex
  \usepackage[T1]{fontenc}
  \usepackage[utf8]{inputenc}
  \usepackage{textcomp} % provide euro and other symbols
\else % if luatex or xetex
  \usepackage{unicode-math}
  \defaultfontfeatures{Scale=MatchLowercase}
  \defaultfontfeatures[\rmfamily]{Ligatures=TeX,Scale=1}
\fi
% Use upquote if available, for straight quotes in verbatim environments
\IfFileExists{upquote.sty}{\usepackage{upquote}}{}
\IfFileExists{microtype.sty}{% use microtype if available
  \usepackage[]{microtype}
  \UseMicrotypeSet[protrusion]{basicmath} % disable protrusion for tt fonts
}{}
\makeatletter
\@ifundefined{KOMAClassName}{% if non-KOMA class
  \IfFileExists{parskip.sty}{%
    \usepackage{parskip}
  }{% else
    \setlength{\parindent}{0pt}
    \setlength{\parskip}{6pt plus 2pt minus 1pt}}
}{% if KOMA class
  \KOMAoptions{parskip=half}}
\makeatother
\usepackage{xcolor}
\IfFileExists{xurl.sty}{\usepackage{xurl}}{} % add URL line breaks if available
\IfFileExists{bookmark.sty}{\usepackage{bookmark}}{\usepackage{hyperref}}
\hypersetup{
  pdftitle={Math 15910: Problem Set 2},
  pdfauthor={Underland, Jake},
  hidelinks,
  pdfcreator={LaTeX via pandoc}}
\urlstyle{same} % disable monospaced font for URLs
\usepackage[margin=1in]{geometry}
\usepackage{graphicx,grffile}
\makeatletter
\def\maxwidth{\ifdim\Gin@nat@width>\linewidth\linewidth\else\Gin@nat@width\fi}
\def\maxheight{\ifdim\Gin@nat@height>\textheight\textheight\else\Gin@nat@height\fi}
\makeatother
% Scale images if necessary, so that they will not overflow the page
% margins by default, and it is still possible to overwrite the defaults
% using explicit options in \includegraphics[width, height, ...]{}
\setkeys{Gin}{width=\maxwidth,height=\maxheight,keepaspectratio}
% Set default figure placement to htbp
\makeatletter
\def\fps@figure{htbp}
\makeatother
\setlength{\emergencystretch}{3em} % prevent overfull lines
\providecommand{\tightlist}{%
  \setlength{\itemsep}{0pt}\setlength{\parskip}{0pt}}
\setcounter{secnumdepth}{-\maxdimen} % remove section numbering

\title{Math 15910: Problem Set 2}
\author{Underland, Jake}
\date{2021-08-09}

\begin{document}
\maketitle

{
\setcounter{tocdepth}{2}
\tableofcontents
}
\hypertarget{exercise-1}{%
\section{Exercise 1}\label{exercise-1}}

\hypertarget{problem-1.4.5}{%
\subsection{Problem 1.4.5}\label{problem-1.4.5}}

\emph{Prove that
\(A \times \emptyset = \emptyset \times A = \emptyset\)}

Let \(A\) and \(B\) be sets. Suppose \(B = \emptyset\). Then, by
definition, there does not exist any \(b \in B\). Since the Cartesian
product of \(A\) and \(B\), \(A \times B\) is defined as the set of the
ordered pairs \((a, b)\) such that \(a \in A\) and \(b \in B\), if there
does not exist any \(b \in B\), then there cannot exist any ordered pair
\((a, b)\). Therefore, there exists no \((a, b) \in A \times B\),
implying that \(A \times B = A \times \emptyset = \emptyset\).
Similarly, \(B \times A\) is defined as the set of the ordered pairs
\((b, a)\) such that \(b \in B\) and \(a \in A\). If there does not
exist any \(b \in B\), then there cannot exist any ordered pair
\((b, a)\), hence \(B \times A\) would be empty. Thus,
\(B \times A = \emptyset \times A = \emptyset\).\\
From the above, we obtain
\(A \times \emptyset = \emptyset \times A = \emptyset\). \hfill \(\Box\)

\hypertarget{exercise-2}{%
\section{Exercise 2}\label{exercise-2}}

\hypertarget{prove-theorem-1.4.7}{%
\subsection{Prove Theorem 1.4.7}\label{prove-theorem-1.4.7}}

\emph{If \(A\) has \(m\) elements and \(B\) has \(n\) elements, then
\(A \times B\) has \(mn\) elements}

Let \(m\) be fixed, and let \(A\) be a set with \(m\) elements. \(B\) is
a set with \(n\) elements. \(m, n \in \mathbb{N}\).\\
If \(n=1\), there is one element \(b \in B\). Then, \(A \times B\) would
be the set of all ordered pairs \((a_i, b)\) for \(1 \leq i \leq m\) and
\(a \in A, b \in B\).\\
\[\underbrace{(a_1, b), (a_2, b), \dots, (a_m, b)}_{m \text{ pairs}}\]
The total number of ordered pairs would be \(m\), and since \(n=1\), the
statement ``\(A \times B\) has \(mn\) elements'' holds true.\\
Suppose the statement is true for \(n = k\). Now assume \(n = k+1\).
Then, the elements in \(A \times B\) would be the combination of pairs
created by \((a_i, b_j)\) where \(1 \leq i \leq m\) and
\(1 \leq j \leq k\), which is equivalent to \(A \times B\) when
\(n = k\), plus the combination of pairs created by \((a_i, b_{k+1})\)
where \(1 \leq i \leq m\). From our supposition, the number of pairs
created by \((a_i, b_j)\) is \(mk\). Furthermore, the number of pairs
created by \((a_{k+1}, b_j)\) is \(m\), as can be seen below:\\
\[\underbrace{(a_1, b_{k+1}), (a_2, b_{k+1}), \dots, (a_m, b_{k+1})}_{m \text{ pairs}}\]
Thus, the total number of elements in \(A \times B\) for \(n=k+1\) would
be \[mk + m = m(k + 1)\] It follows that when we suppose the statement
to be true for \(n = k\), it holds true for \(n = k+1\). Therefore, we
have shown by induction that If \(A\) has \(m\) elements and \(B\) has
\(n\) elements, then \(A \times B\) has \(mn\) elements.

\hfill \(\Box\)

\hypertarget{exercise-3}{%
\section{Exercise 3}\label{exercise-3}}

\hypertarget{problem-1.4.9}{%
\subsection{Problem 1.4.9}\label{problem-1.4.9}}

\emph{Prove that if \(A_1\) has \(k_1\) elements, \(A_2\) has \(k_2\)
elements, \ldots, \(A_n\) has \(k_n\) elements, show that
\(|A_1 \times A_2 \times \dots \times A_n| = |A_1||A_2|\dots|A_n|=k_1k_2\dots k_n\).}

When \(n = 2\), we have already proven in the previous exercise that
\(|(A_1 \times A_2)| = |A_1||A_2| =k_1k_2\). Thus, the above statement
holds.\\
Now suppose that the statement holds for \(n = j\). When \(n = j+1\),
\[A_1 \times A_2 \times \dots \times A_{j} \times A_{j+1}\]
\[= (A_1 \times A_2 \times \dots \times A_j) \times A_{j+1}\] Here, we
substitute \(B = A_1 \times A_2 \times \dots \times A_j\) and treat
\(B\) as a single set. From our supposition,
\(|B| = |A_1 \times A_2 \times \dots \times A_j|=|A_1||A_2|\dots|A_j|=k_1k_2\dots k_j\).
Then, \[|A_1 \times A_2 \times \dots \times A_{j} \times A_{j+1}|\]
\[= |B \times A_{j+1}|\] From the previous exercise,
\[|B \times A_{j+1}|\] \[= |B||A_{j+1}| \] \[= k_1k_2\dots k_j k_{j+1}\]
Thus, the statement holds.\\
Therefore, we have proven by induction that if \(A_1\) has \(k_1\)
elements, \(A_2\) has \(k_2\) elements, \ldots, \(A_n\) has \(k_n\)
elements, then
\(|A_1 \times A_2 \times \dots \times A_n| = |A_1||A_2|\dots|A_n|=k_1k_2\dots k_n\).

\hfill \(\Box\)

\hypertarget{exercise-4}{%
\section{Exercise 4}\label{exercise-4}}

\hypertarget{problem-1.4.10}{%
\subsection{Problem 1.4.10}\label{problem-1.4.10}}

\emph{i. If \(A\) and \(B\) are finite sets and
\(A \cap B = \emptyset\), show that \(|A\cup B| = |A| + |B|\)}

Suppose that \(|A| = m\), \(|B| = n\) and \(A \cap B = \emptyset\).
Then, the elements of \(A\) can be written as \(a_1, a_2, \dots, a_m\)
and the elements of \(B\) can be written as \(b_1, b_2, \dots, b_n\). If
\(A \cap B = \emptyset\), then by definition, \(a_i \neq b_j\) for all
\(1 \leq i \leq m\), \(1 \leq j \leq n\). It follows that the union of
the two sets would contain the elements
\[\underbrace{a_1, a_2, \dots, a_m}_{m}, \underbrace{b_1, b_2, \dots, b_n}_{n}\]
without overlap. The number of elements in the union would be sum of the
number of elements in \(A\) which is \(|A| = m\) and the number of
elements in \(B\) which is \(|B| = n\).\\
Therefore, if \(A\) and \(B\) are finite sets and
\(A \cap B = \emptyset\), then \(|A\cup B| = |A| + |B|\).
\hfill \(\Box\)

\emph{ii. If \(A\) and \(B\) are finite sets, show that
\(|A\cup B| = |A| + |B| - |A\cap B|\)}

Suppose that \(|A| = m\), \(|B| = n\), and \(|A \cap B|=x\). Then, the
elements of \(A\) can be written as
\(\{a_1, a_2, \dots, a_{m-x}, c_1, c_2, ..., c_x\}\) and the elements of
\(B\) can be written as
\(\{b_1, b_2, \dots, b_{n-x},c_1, c_2, ..., c_x\}\) where
\(a \in A \backslash B\), \(b \in B \backslash A\), \(c \in A \cap B\).
Then,
\[A\cup B = \{\underbrace{a_1, a_2, \dots, a_{m-x}}_{m-x}, \underbrace{b_1, b_2, \dots, b_{n-x}}_{n-x}, \underbrace{c_1, c_2, ..., c_x}_x\}\]
\[\Rightarrow |A \cup B| = (m - x) + (n - x) + x\] \[= m + n -x\]
\[= |A| + |B| - |A\cap B|\] Thus, we have shown that if \(A\) and \(B\)
are finite sets, \(|A\cup B| = |A| + |B| - |A\cap B|\). \hfill \(\Box\)

\hypertarget{exercise-5}{%
\section{Exercise 5}\label{exercise-5}}

\emph{Prove that, for all \(n \in \mathbb{N}\),}
\[1^2 + 2^2 + \dots + n^2 = \frac{n(n+1)(2n+1)}{6}\]

Start with \(n=1\). Then,
\[1^2 =1,  \frac{1(1+1)(2*1+1)}{6} = \frac{6}{6} = 1\]
\[\Rightarrow 1^2 = \frac{1(1+1)(2*1+1)}{6} \] Thus, the statement
holds.\\
Next, suppose the statement is true for \(n = k\). Now, let's consider
\(n = k+1\). We begin with the left hand side of the equation.
\[1^2 + 2^2 + \dots + k^2 + (k+1)^2\]
\[= (1^2 + 2^2 + \dots + k^2) + (k+1)^2\] From our assumption,
\[=\frac{k(k+1)(2k+1)}{6} + (k+1)^2\]
\[=\frac{2k^3 + 3k^2 + k}{6} + \frac{6k^2 + 12k + 6}{6}\]
\[= \frac{2k^3 + 9k^2 + 13k + 6}{6}\]
\[=\frac{(k+1)(2k^2 + 7k + 6)}{6}\] \[=\frac{(k+1)(k+2)(2k + 3)}{6}\]
\[=\frac{(k+1)\{(k+1)+1\}\{2(k + 1)+1\}}{6}\] Above, we have derived the
right hand side from the left hand side of the equation. Hence, the
statement is true for \(n = k+1\) given that the statement is true for
\(n = k\).\\
Therefore, we have proven by induction that
\[1^2 + 2^2 + \dots + n^2 = \frac{n(n+1)(2n+1)}{6}\] \hfill \(\Box\)

\hypertarget{exercise-6}{%
\section{Exercise 6}\label{exercise-6}}

\hypertarget{problem-1.7.9}{%
\subsection{Problem 1.7.9}\label{problem-1.7.9}}

\emph{Suppose that \(A\) is a set with \(n\) elements, \(B\) is a set
with \(m\) elements, and \(n>m\). If \(f: A \rightarrow B\) is a
function, there are at least two distinct elements of \(A\) that
correspond to the same element of \(B\).}

We take \(|A| = n\), \(|B| = m\) and \(n>m\) as given. If
\(f: A \rightarrow B\) is a function, suppose that there is at most one
element of \(A\) that corresponds to one element of \(B\). Since a
function that maps from \(A\) to \(B\) is defined as a subset of
\(A \times B\) such that each element of A occurs exactly once as the
first coordinate, if there is at most one element of \(A\) that
corresponds to one element of \(B\), the subset can be written as
\({(a_i, b_i) \text{ for }1 \leq i \leq m}\). It is clear that the
number of elements in this subset is \(m\). By definition of a function,
each element of \(A\) occurs exactly once as the first coordinate, so
the number of elements in \(A\) is equal to the number of elements in
the subset, and \(n=m\). This contradicts \(n>m\). Thus, if \(n>m\) and
if \(f: A \rightarrow B\) is a function, there are at least two distinct
elements of \(A\) that correspond to the same element of \(B\).

\hfill \(\Box\)

\hypertarget{exercise-7}{%
\section{Exercise 7}\label{exercise-7}}

\hypertarget{problem-1.7.15}{%
\subsection{Problem 1.7.15}\label{problem-1.7.15}}

\hypertarget{i.-fmathbbn-rightarrow-mathbbn-fn-2n.}{%
\subsubsection{\texorpdfstring{i.
\(f:\mathbb{N} \rightarrow \mathbb{N}, f(n) = 2n.\)}{i. f:\textbackslash mathbb\{N\} \textbackslash rightarrow \textbackslash mathbb\{N\}, f(n) = 2n.}}\label{i.-fmathbbn-rightarrow-mathbbn-fn-2n.}}

The above function is injective\\
Proof: Take \(x_1, x_2 \in \mathbb{N}\). Suppose that
\(f(x_1) = f(x_2)\). Then, \[2x_1 = 2x_2\]
\[ \Longrightarrow x_1 = x_2\]\\
Since \(f(x_1) = f(x_2)\Longrightarrow x_1 = x_2\), \(f\) is
injective.\\
Next, take \(y \in \mathbb{N}\). We want to show that \(y = f(x)\) for
some \(x \in \mathbb{N}\).\\
\[y = f(x)\] \[\Longrightarrow y = 2x\]
\[\Longrightarrow x = \frac{y}{2}\] However, if \(y\) is an odd number,
there is no \(x \in \mathbb{N}\) that satisfies the above. Therefore,
\(f\) is not surjective.\\
Hence, \(f\) is injective.

\hypertarget{ii.-fmathbbz-rightarrow-mathbbz-fn-n6.}{%
\subsubsection{\texorpdfstring{ii.
\(f:\mathbb{Z} \rightarrow \mathbb{Z}, f(n) = n+6.\)}{ii. f:\textbackslash mathbb\{Z\} \textbackslash rightarrow \textbackslash mathbb\{Z\}, f(n) = n+6.}}\label{ii.-fmathbbz-rightarrow-mathbbz-fn-n6.}}

The above function is bijective.\\
Proof: Take \(x_1, x_2 \in \mathbb{Z}\). Suppose that
\(f(x_1) = f(x_2)\). Then, \[x_1 + 6= x_2 + 6\]
\[ \Longrightarrow x_1 = x_2\]\\
Since \(f(x_1) = f(x_2)\Longrightarrow x_1 = x_2\), \(f\) is
injective.\\
Next, take \(y \in \mathbb{Z}\). We want to show that \(y = f(x)\) for
some \(x \in \mathbb{N}\).\\
\[y = f(x)\] \[\Longrightarrow y = x + 6\] \[\Longrightarrow x = y - 6\]
Since \(\mathbb{Z}\) is a ring and \(y, 6 \in \mathbb{Z}\),
\(y-6 \in \mathbb{Z}\). Thus, we have found that for all
\(y \in \mathbb{Z}\), \(y = f(x)\) for some \(x \in \mathbb{Z}\).\\
Hence, \(f\) is bijective.

\hypertarget{iii.-fmathbbn-rightarrow-mathbbq-fn-n.}{%
\subsubsection{\texorpdfstring{iii.
\(f:\mathbb{N} \rightarrow \mathbb{Q}, f(n) = n.\)}{iii. f:\textbackslash mathbb\{N\} \textbackslash rightarrow \textbackslash mathbb\{Q\}, f(n) = n.}}\label{iii.-fmathbbn-rightarrow-mathbbq-fn-n.}}

The above function is injective.\\
Proof: Take \(x_1, x_2 \in \mathbb{N}\). Suppose that
\(f(x_1) = f(x_2)\). Then, \[x_1= x_2\] Since
\(f(x_1) = f(x_2)\Longrightarrow x_1 = x_2\), \(f\) is injective.\\
Next, take \(y \in \mathbb{Q}\). We want to show that \(y = f(x)\) for
some \(x \in \mathbb{N}\).\\
\[y = f(x)\] \[\Longrightarrow y = x\] \[\Longrightarrow x = y\]
However, the above does not hold for any
\(y \in \mathbb{Q} \backslash \mathbb{N}\). Therefore, \(f\) is not
surjective.\\
Hence, \(f\) is injective.

\hypertarget{v.-fmathbbr-rightarrow-mathbbn-fx-text-the-third-digit-of-x-text-after-the-decimal.}{%
\subsubsection{\texorpdfstring{v.
\(f:\mathbb{R} \rightarrow \mathbb{N}, f(x) = \text{ the third digit of } x \text{ after the decimal.}\)}{v. f:\textbackslash mathbb\{R\} \textbackslash rightarrow \textbackslash mathbb\{N\}, f(x) = \textbackslash text\{ the third digit of \} x \textbackslash text\{ after the decimal.\}}}\label{v.-fmathbbr-rightarrow-mathbbn-fx-text-the-third-digit-of-x-text-after-the-decimal.}}

The above function is neither injective nor surjective.\\
Proof: Take \(x_1 = 0.125\) and \(x_2 = 5.5555\). Then,
\(f(x_1) = f(x_2)\). However, \(x_1 \neq x_2\). Thus,
\(f(x_1) = f(x_2) \nRightarrow x_1 = x_2\), and \(f\) is not
injective.\\
Next, take \(y \in \mathbb{N}\). We want to show that \(y = f(x)\) for
some \(x \in \mathbb{N}\).\\
\[y = f(x)\]
\[\Longrightarrow y = \text{the third digit of } x \text{ after the decimal.}\]
\[\Longrightarrow x = \text{Any real number whose decimal expansion has } y \text{ in the third digit after the decimal}\]
However, the above does not hold for any \(y > 9\), as the digit must be
a natural number from 1 to 9. Therefore, \(f\) is not surjective.\\
Hence, \(f\) is neither injective nor surjetive.

\hypertarget{exercise-8}{%
\section{Exercise 8}\label{exercise-8}}

\hypertarget{problem-1.7.22}{%
\subsection{Problem 1.7.22}\label{problem-1.7.22}}

\emph{Show that} \(f:\mathbb{N} \rightarrow \mathbb{Z}\)
\[f(n) = \begin{cases} \frac{n}{2}, \text{       if }n \text{ is even} \\ \frac{1 -n}{2}, \text{       if }n \text{ is odd.} \end{cases}\]
\emph{is a bijection}

~~~~Take \(x_1, x_2 \in \mathbb{N}\). Suppose that \(f(x_1) = f(x_2)\).
We consider the following three cases:\\
\hspace*{0.333em}\hspace*{0.333em}\hspace*{0.333em}\hspace*{0.333em}1.
If both \(x_1\) and \(x_2\) are even. Then, \(x_1\) and \(x_2\) can be
expressed as \(2m_1, 2m_2\) for some \(m_1, m_2 \in \mathbb{N}\). Thus,
\[f(x_1) = f(x_2)\]

\[\Longrightarrow \frac{2m_1}{2} = \frac{2m_2}{2}\]
\[\Longrightarrow m_1 = m_2\] \[\Longrightarrow x_1 = x_2\] ~~~~2. If
both \(x_1\) and \(x_2\) are odd. Then, \(x_1\) and \(x_2\) can be
expressed as \(2m_1 - 1, 2m_2 - 1\) for some
\(m_1, m_2 \in \mathbb{N}\). Thus, \[f(x_1) = f(x_2)\]

\[\Longrightarrow \frac{1 - (2m_1 -1)}{2} = \frac{1-(2m_2-1)}{2}\]
\[\Longrightarrow \frac{2 - 2m_1}{2} = \frac{2-2m_2}{2}\]
\[\Longrightarrow 1 - m_1 = 1-m_2\] \[\Longrightarrow m_1 = m_2\]
\[\Longrightarrow x_1 = x_2\]\\
\hspace*{0.333em}\hspace*{0.333em}\hspace*{0.333em}\hspace*{0.333em}3.
If \(x_1\) is even and \(x_2\) is odd. Then, \(x_1\) and \(x_2\) can be
expressed as \(2m_1, 2m_2 - 1\) for some \(m_1, m_2 \in \mathbb{N}\).
Thus, \[f(x_1) = f(x_2)\]

\[\Longrightarrow \frac{2m_1 }{2} = \frac{1-(2m_2-1)}{2}\]
\[\Longrightarrow m_1 = 1 -m_2\] However, there are no
\(m_1, m_2 \in \mathbb{N}\) that satisfy the above expression, since
\(m_1 \geq 1, 1-m_2 \leq 0\). Similarly, if \(x_1\) is odd and \(x_2\)
is even, \(f(x_1) \neq f(x_2)\).\\
\hspace*{0.333em}\hspace*{0.333em}\hspace*{0.333em}\hspace*{0.333em}From
3., we know that if \(f(x_1) = f(x_2)\), then either both \(x_1\) and
\(x_2\) are odd or both are even. From 1. and 2., we know that if are
odd or even, \(f(x_1) = f(x_2)\) implies that \(x_1 = x_2\). Thus, \(f\)
is an injection.

~~~~Now, take \(y \in \mathbb{Z}\). We want to show that \(y = f(x)\)
for some \(x \in \mathbb{N}\).\\
If \(y > 0\):\\
\[y = f(x)\] \[\Longrightarrow y = \frac{x}{2}\]
\[\Longrightarrow x = 2y\] Since \(y \in \mathbb{Z}, y > 0\), we get
that \(2y \in \mathbb{N}\) and \(2y\) is even. From here, we obtain that
for all \(y>0\), \(y = f(x)\) for some even \(x \in \mathbb{N}\).\\
If \(y \leq 0\):\\
\[y = f(x)\] \[\Longrightarrow y = \frac{1 - x}{2}\]
\[\Longrightarrow x = 1 - 2y\] Since \(y \in \mathbb{Z}, y \leq 0\), we
get that \(1 - 2y \in \mathbb{N}\), and \(1-2y\) is odd. From here, we
obtain that for all \(y\leq0\), \(y = f(x)\) for some odd
\(x \in \mathbb{N}\).\\
\hspace*{0.333em}\hspace*{0.333em}\hspace*{0.333em}\hspace*{0.333em}Together,
we have shown that for all \(y \in \mathbb{Z}\), \(y = f(x)\) for some
\(x \in \mathbb{N}\). Thus, \(f\) is surjective.\\
\hspace*{0.333em}\hspace*{0.333em}\hspace*{0.333em}\hspace*{0.333em}Since
\(f\) is both injective and surjective, \(f\) is bijective.

\hfill \(\Box\)

\end{document}
