% Options for packages loaded elsewhere
\PassOptionsToPackage{unicode}{hyperref}
\PassOptionsToPackage{hyphens}{url}
%
\documentclass[
]{article}
\usepackage{lmodern}
\usepackage{amssymb,amsmath}
\usepackage{ifxetex,ifluatex}
\ifnum 0\ifxetex 1\fi\ifluatex 1\fi=0 % if pdftex
  \usepackage[T1]{fontenc}
  \usepackage[utf8]{inputenc}
  \usepackage{textcomp} % provide euro and other symbols
\else % if luatex or xetex
  \usepackage{unicode-math}
  \defaultfontfeatures{Scale=MatchLowercase}
  \defaultfontfeatures[\rmfamily]{Ligatures=TeX,Scale=1}
\fi
% Use upquote if available, for straight quotes in verbatim environments
\IfFileExists{upquote.sty}{\usepackage{upquote}}{}
\IfFileExists{microtype.sty}{% use microtype if available
  \usepackage[]{microtype}
  \UseMicrotypeSet[protrusion]{basicmath} % disable protrusion for tt fonts
}{}
\makeatletter
\@ifundefined{KOMAClassName}{% if non-KOMA class
  \IfFileExists{parskip.sty}{%
    \usepackage{parskip}
  }{% else
    \setlength{\parindent}{0pt}
    \setlength{\parskip}{6pt plus 2pt minus 1pt}}
}{% if KOMA class
  \KOMAoptions{parskip=half}}
\makeatother
\usepackage{xcolor}
\IfFileExists{xurl.sty}{\usepackage{xurl}}{} % add URL line breaks if available
\IfFileExists{bookmark.sty}{\usepackage{bookmark}}{\usepackage{hyperref}}
\hypersetup{
  pdftitle={Math 15910: Problem Set 7},
  pdfauthor={Underland, Jake},
  hidelinks,
  pdfcreator={LaTeX via pandoc}}
\urlstyle{same} % disable monospaced font for URLs
\usepackage[margin=1in]{geometry}
\usepackage{graphicx,grffile}
\makeatletter
\def\maxwidth{\ifdim\Gin@nat@width>\linewidth\linewidth\else\Gin@nat@width\fi}
\def\maxheight{\ifdim\Gin@nat@height>\textheight\textheight\else\Gin@nat@height\fi}
\makeatother
% Scale images if necessary, so that they will not overflow the page
% margins by default, and it is still possible to overwrite the defaults
% using explicit options in \includegraphics[width, height, ...]{}
\setkeys{Gin}{width=\maxwidth,height=\maxheight,keepaspectratio}
% Set default figure placement to htbp
\makeatletter
\def\fps@figure{htbp}
\makeatother
\setlength{\emergencystretch}{3em} % prevent overfull lines
\providecommand{\tightlist}{%
  \setlength{\itemsep}{0pt}\setlength{\parskip}{0pt}}
\setcounter{secnumdepth}{-\maxdimen} % remove section numbering

\title{Math 15910: Problem Set 7}
\author{Underland, Jake}
\date{2021-08-09}

\begin{document}
\maketitle

{
\setcounter{tocdepth}{2}
\tableofcontents
}
\hypertarget{exercise-1}{%
\section{Exercise 1}\label{exercise-1}}

\hypertarget{problem-3.2.9}{%
\subsection{Problem 3.2.9}\label{problem-3.2.9}}

\textit{Suppose $\lim _{n \to \infty} x_n = x $ and $\lim _{n \to \infty} y_n = y $}

\begin{enumerate}
\item \textit{Show that $\lim _{n \to \infty} (x_n + y_n) = x + y$}  
  
Proof. From the supposition, we know that for all $\epsilon >0$ there exists an $N \in \mathbb{N}$ such that for all $n>N$,
\[\begin{aligned}|x_n - x| &< \epsilon  \\ |y_n - y| &< \epsilon\end{aligned}\]
Let $\epsilon = \frac{\epsilon'}{2}$. Then, 
\[\begin{aligned}|x_n - x| &< \frac{\epsilon'}{2} \\ |y_n - y| &< \frac{\epsilon'}{2}\end{aligned}\] 
From triangle inequality, 
\[\begin{aligned}|x_n + y_n - x - y| \leq |x_n - x| + |y_n - y| &< \frac{\epsilon'}{2}+ \frac{\epsilon'}{2} \\ |x_n + y_n - (x + y)|&< \epsilon' \end{aligned}\]
This holds for all $\epsilon' > 0$. Thus, $x + y$ is the limit. 

\hfill $\Box$  
 
\item \textit{Show that if for all $n \in \mathbb{N} y_n ̸\neq 0$ and $y \neq 0$, then $\lim_{n\to \infty}(\frac{1}{y_n}) = \frac{1}{y}$.}  
  
Proof. Since $\lim _{n \to \infty} y_n = y$, it is obvious that $\lim _{n \to \infty} |y_n| = |y|$. Then, for all $\epsilon > 0$, there exists an $N$ such that for all $n > N$, $|y_n|$ becomes arbitrarily close to $|y|$. Thus, we can say with certainty that there exists $N_1$ such that for all $n>N_1$, $|y_n| > \frac{|y|}{2}$. This implies $\frac{1}{|y_n|} < \frac{2}{|y|}$.  
Further, let $\epsilon = \epsilon' \frac{|y|^2}{2}$. Then, there also exists $N_2$ such that for all $n > N_2$, $|y_n - y| < \epsilon$.  
Now, let us set $N=\max \{ N_1, N_2\}$. Then, 
\[\begin{aligned}&|\frac{1}{y_n} - \frac{1}{y}| 
\\ = &|\frac{y - y_n}{y_n y}| 
\\ = &\frac{|y_n - y|}{|y_n||y|} 
\\ < &\epsilon' \frac{|y|^2}{2} \cdot \frac{2}{|y|} \cdot \frac{1}{|y|}
\\ = &\epsilon'
\end{aligned}\]
Therefore, for all $\epsilon'>0$, there exists $N \in \mathbb{N}$ such that for all $n > N$, $|\frac{1}{y_n} - \frac{1}{y}| < \epsilon'$. Thus,  $\lim_{n\to \infty}(\frac{1}{y_n}) = \frac{1}{y}$.

\hfill $\Box$ 
\end{enumerate}

\hypertarget{exercise-2}{%
\section{Exercise 2}\label{exercise-2}}

\textit{Let $m \in \mathbb{N}$. Prove that $(x_n)_{n \in \mathbb{N}}$ converges iff $(x_{m+n})_{n \in \mathbb{N}}$ converges. Moreover, show that $\lim_{n\to\infty} x_n = \lim_{n\to\infty} x_{m+n}$}

~~~~Proof. If \((x_n)_{n \in \mathbb{N}}\) converges to \(x\), then we
know that for all \(\epsilon >0\) there exists an \(N \in \mathbb{N}\)
such that for all \(n>N\), \(|x_n - x|<\epsilon\). Since \(m + n > n\),
\((x_{m+n})\) also converges to the same limit \(x\).\\
\hspace*{0.333em}\hspace*{0.333em}\hspace*{0.333em}\hspace*{0.333em}If
\((x_{m+n})_{n \in \mathbb{N}}\) converges to \(x\), then for all
\(\epsilon >0\) there exists an \(N_1 \in \mathbb{N}\) such that for all
\(m + n>N_1\), \(|x_{m+n} - x|<\epsilon\). Then, take \(N_2 = N_1 + m\).
For all \(n + m> N_2, \space n+m > N_1 + m > N_1\). Furthermore,
\(n+m > N_1 + m \implies n > N_1\). Thus, \(|x_{n} - x|<\epsilon\). Such
an \(N_2\) exists for all \(\epsilon\), given that
\((x_{m+n})_{n \in \mathbb{N}}\) converges. Thus,
\((x_n)_{n \in \mathbb{N}}\) converges iff
\((x_{m+n})_{n \in \mathbb{N}}\), and their limits are equivalent.

\hfill \(\Box\)

\hypertarget{exercise-3}{%
\section{Exercise 3}\label{exercise-3}}

\textit{Show that $(x_n)_{n\in \mathbb{N}}$ converges to $L$ if and only if every subsequence of $(x_n)_{n\in \mathbb{N}}$ converges to $L$}

~~~~Proof. Let \(b_n\) be a subsequence of \(x_n\). If

\hfill \(\Box\)

\hypertarget{exercise-4}{%
\section{Exercise 4}\label{exercise-4}}

\emph{Show that every bounded open interval
\((a,b) \subseteq \mathbb{R}\) can be described as
\(\{x\in \mathbb{R}||x−x_0|<\epsilon\}\) for some value
\(x_0 \in \mathbb{R}\) and some \(\epsilon>0\). What are the values of
\(x_0\) and \(\epsilon\) in terms of \(a\) and \(b\)? And conversely,
given \(x_0\) and \(ε\), what are the endpoints of the interval
\(\{x\in \mathbb{R}||x−x_0|<\epsilon\}\)?}

Proof. In this definition, \(x_0\) denotes the median point between
\(a\) and \(b\), or \(\frac{a + b}{2}\), and \(\epsilon\) denotes half
of the length of the interval \(\frac{|b - a|}{2}\).
\(\{x\in \mathbb{R}||x−x_0|<\epsilon\}\) says that for every \(x \in R\)
such that the distance between \(x\) and \(x_0\), the median point of
the interval \((a, b)\), measured by \(|x−x_0|\), must be less than
\(\epsilon\) which is the distance from this median point to \(a\) or
\(b\). Thus the endpoints of this interval can be described as
\(x_0 \pm \epsilon\), and the interval can be restated as
\((x_0 - \epsilon, x_0 + \epsilon)\) or the symmetric neighborhood of
\(x_0\).

\hfill \(\Box\)

\hypertarget{exercise-5}{%
\section{Exercise 5}\label{exercise-5}}

\hypertarget{problem-3.4.3}{%
\subsection{Problem 3.4.3}\label{problem-3.4.3}}

\emph{Suppose that \(I\) is a subset of \(\mathbb{R}\). Show that \(I\)
is an interval if and only if for all \(a, b \in I\), with \(a \leq b\),
the closed interval \([a, b] \subseteq I\).}

~~~~Proof. Let \(x, y\) be the endpoints of \(I\), where \(x < y\).
Suppose \(I\) is an interval. Take \(a, b \in I\). Then, \(x \leq a\)
and \(b \leq y\), or else \(a, b \notin I\). Because, all values between
\(x\) and \(y\) are in \(I\), any value between \(a\) and \(b\) is also
between \(x\) and \(y\) and is therefore inside \(I\). From our
supposition, \(a, b \in I\). Thus, \([a, b] \subseteq I\).\\
\hspace*{0.333em}\hspace*{0.333em}\hspace*{0.333em}\hspace*{0.333em}Now,
take the statement if for all \(a, b \in I\) such that \(a \leq b\),
\([a, b] \subseteq I\), then \(I\) is an interval. We prove the
converse. Suppose \(I\) is not an interval. This means \(I\) cannot be
written in any of the 10 categories described in Definition 3.4.1 of
Sally. This means there is a discontinuity between 2 values in \(I\), or
2 values \(c, d \in I\) where \(c < d\) such that
\(\frac{c + d}{2} \notin I\). Then, it follows that for some
\(a, b \in I\) such that \(a \leq b\), \([a, b] \nsubseteq I\). Thus,
the converse is true, and we have proven the statement.

\hfill \(\Box\)

\hypertarget{exercise-6}{%
\section{Exercise 6}\label{exercise-6}}

\emph{Let \(I_n = [a_n, b_n]\) for \(n \in \mathbb{N}\) be a nested
collection of intervals. Suppose that
\(\inf \{b_n - a_n|n\in \mathbb{N}\} = 0\). Show that the number
\(\xi \in \cap^{\infty}_{n=1} I_n\) is unique.}

~~~~Proof. Suppose that \(\inf \{b_n - a_n|n\in \mathbb{N}\} = 0\).
Then, for one of the intervals \(I_\xi\), \(a_n = b_n\). Thus, the
interval is a point, or unique value \(\xi\) such that
\(a_n = b_n = \xi\). By the Nested Interval Theorem, we know that
\(\cap^{\infty}_{n=1} I_n \neq \emptyset\)/. Therefore, there must be at
least one value that is found in every interval. However, since we know
that one of these intervals only contains one unique value \(\xi\), for
any value \(i \in I_i\) to be in \(\cap^{\infty}_{n=1} I_n\),
\(i \in I_\xi \implies i = \xi\). Thus,
\(\xi \in \cap^{\infty}_{n=1} I_n\) is unique.

\hfill \(\Box\)

\hypertarget{exercise-7}{%
\section{Exercise 7}\label{exercise-7}}

\hypertarget{problem-3.6.5}{%
\subsection{Problem 3.6.5}\label{problem-3.6.5}}

\emph{Show that the limit of a convergent sequence is unique.}

~~~~Proof. Let us denote this sequence by \(a_n\) and its limits
\(L, M\). Suppose \(L \neq M\). Then, for every \(\epsilon >0\), there
exists \(N_L, N_M \in \mathbb{N}\) such that if \(n > N_L\),
\(|a_n - L| < \epsilon\) and if \(n > N_M\), \(|a_n - M| < \epsilon\).
Let \(\epsilon = \frac{|L - M|}{4}\), and \(n > \max \{N_L, N_M\}\).
Then, from trianble inequality in theorem 3.6.2,\\
\[\begin{aligned} |L - M| \leq |L - a_n| &+ |a_n - M| < 2 \epsilon 
\\ \implies |L - M| \leq |L - a_n| &+ |a_n - M| < \frac{|L - M|}{2} 
\\ \implies |L - M| &< \frac{|L - M|}{2} \end{aligned}\] This is
impossible, since \(|L - M| > 0\). Thus, \(L = M\), and the limit of a
sequence is unique.

\hfill \(\Box\)

\hypertarget{exercise-8}{%
\section{Exercise 8}\label{exercise-8}}

\emph{Show that the sequence \(a_n = (-1)^n\) is divergent}

~~~~Proof. Take \(a \in \mathbb{R}\) and \(\epsilon = 1\). Then, for all
\(N \in \mathbb{N}\), take \(n = 2N > N\). Then,
\[\begin{aligned}|a_n - a| &= |(-1)^n - a| \\ 
&=|(-1)^{2N} - a| \\
&= |1 - a|\end{aligned}\] If \(a \leq 0\) or \(a \geq 2\), then
\(|1-a| \geq \epsilon\), and \(a\) is not a limit. If \(0 < a < 2\), we
change \(n\) to be \(2N + 1 > 2N > N\). Then,
\[\begin{aligned}|a_n - a| &= |(-1)^n - a| \\ 
&=|(-1)^{2N+1} - a| \\
&= |-1 - a| \geq 1 = \epsilon \end{aligned}\] Thus, for all
\(a \in \mathbb{R}\), there exists \(\epsilon > 0\) such that for all
\(N \in \mathbb{N}\), there is an \(n > N\) such that
\(|a_n - a| \geq \epsilon\). This shows that the sequence is divergent.

\hfill \(\Box\)

\hypertarget{bonus}{%
\section{Bonus}\label{bonus}}

\emph{Use the nested interval theorem to obtain a new proof of the fact
that R is uncountable.}

~~~~Proof. We already know there is a bijection between \((0, 1)\) and
\(\mathbb{R}\), so we will prove that \((0, 1)\) is uncountable.\\
\hspace*{0.333em}\hspace*{0.333em}\hspace*{0.333em}\hspace*{0.333em}Suppose
it is countable. Then, the elements of \((0, 1)\) are subscriptable as
\(\{a_1, a_2, \dots, a_n, \dots\}\). Since \((0, 1)\) is a bounded
interval in \(\mathbb{R}\), it contains a nested sequence of closed
bounded intervals in \(\mathbb{R}\). Let these intervals \(I_n\) be such
that \(a_n \notin I_n\). From the nested interval theorem,
\(\cap ^\infty _{n=1} I_n\) is nonempty. Then, there exists \(x_i\) such
that \(x_i \in \cap ^\infty _{n=1} I_n\), or \(x_i\) is in every single
interval \(I_n\). However, this is a contradiction since the set of
\(I_{n > i}\) do not include \(x_i\) by definition. Therefore,
\((0, 1)\) is not countable, and concomitantly, \(\mathbb{R}\) is
uncountable.

\hfill \(\Box\)

\end{document}
